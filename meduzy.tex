\documentclass[12pt]{report}
\usepackage{multicol}
\usepackage[czech]{babel}
\usepackage{multicol}
\usepackage[dvipsnames]{xcolor}
\usepackage{a4wide}
\usepackage{booktabs}
\usepackage{longtable}
\usepackage{algorithm}
\usepackage{threeparttable}
\usepackage{algorithmic}
\usepackage{amsthm}
\usepackage{amsmath}
\usepackage{array}
\usepackage{thmtools}
\usepackage{mathtools}
\usepackage{amsmath, amssymb, amsfonts}
\usepackage[all]{xypic}
\usepackage{graphicx}
\usepackage{fancyhdr}
\usepackage{caption}
\usepackage{mathalfa}
\usepackage{verse}
\usepackage{extarrows}
\usepackage{enumerate}
\renewcommand{\labelenumi}{(\roman{enumi})} 
\usepackage{tabularx}
\usepackage{wrapfig}
\usepackage{color}
\usepackage{array}
\usepackage{textcomp}
\usepackage{siunitx}
\usepackage{tikz-cd}
\usetikzlibrary{arrows}
\usepackage{pgfplots}
\usetikzlibrary{babel}
\usepackage{epsfig}
\PassOptionsToPackage{hyphens}{url}\usepackage{hyperref}
\hypersetup{colorlinks,
citecolor=RoyalBlue
}
\usepackage{url}
\usepackage{tabularx}
\usetikzlibrary{intersections}
\usepackage[metapost, mplabels,truebbox,clip]{mfpic}
\allowdisplaybreaks
\usepackage{scrextend}
\makeatletter
\newcommand\figcaption{\def\@captype{figure}\caption}
\makeatother

\renewcommand{\sectionmark}[1]{ \markright{-- \thesection.\ #1}{}}

\fancypagestyle{plain}{%
\fancyhf{}
\lhead[L]{}
\chead{}
\rhead[R]{\nouppercase\leftmark}
\lfoot{}
\cfoot{}
\rfoot{\thepage}
\renewcommand{\headrulewidth}{0.4pt}}

\pagestyle{fancy}
\fancyhf{}
\lhead[L]{}
\chead{}
\rhead[R]{\nouppercase\leftmark}
\lfoot{}
\cfoot{}
\rfoot{\thepage}
\begin{document}
\newcommand{\ZZ}{{\mathbb{Z}}}
\newcommand{\cyc}[1]{{\langle #1 \rangle}}


\newtheorem{veta}{Věta}[section]
%\newtheorem{definice}{Definice}[section]
%\newtheorem{dusledek}{Důsledek}[section]
%\newtheorem{lemma}{Lemma}[section]


\theoremstyle{de}
\newtheorem{de}{Definition}[section]
\newtheorem{dusledek}[veta]{Důsledek}
\newtheorem{lemma}[veta]{Lemma}
\newtheorem*{lemma*}{Lemma}


\theoremstyle{definition}
\newtheorem{priklad}[veta]{Příklad}
\newtheorem{definice}[veta]{Definice}
\newtheorem{problem}[veta]{Problém}
\newtheorem{znaceni}[veta]{Značení}
\newtheorem*{umluva}{Úmluva}
\newtheorem*{poznamka}{Poznámka}
\newtheorem{dfn}[veta]{Definition}

\floatname{algorithm}{Algoritmus}

\setlength{\parindent}{2ex}

\def\Tr{\operatorname{Tr}}
\def\N{\operatorname{N}}
\def\SL{\operatorname{SL}}
\def\nsd{\operatorname{nsd}}
\def\End{\operatorname{End}}
\def\id{\operatorname{id}}
\def\char{\operatorname{char}}
\def\ker{\operatorname{ker}}
\def\Aut{\operatorname{Aut}}
\def\Gal{\operatorname{Gal}}
\def\Fix{\operatorname{Fix}}
\def\c{\operatorname{\mathbb{C}}}
\def\R{\operatorname{\mathbb{R}}}
\def\q{\operatorname{\mathbb{Q}}}
\def\e{\operatorname{\mathcal{V}}}
\def\z{\operatorname{\mathbb{Z}}}
\def\n{\operatorname{\mathbb{N}}}
\def\s{\operatorname{\subseteq}}
\def\w{\operatorname{\zeta}}
\def\fii{\operatorname{\varphi}}
\def\o{\operatorname{\mathcal{O}}}
\def\I{\operatorname{\mathcal{I}}}
\def\J{\operatorname{\mathcal{J}}}
\def\P{\operatorname{\mathcal{P}}}
\def\pn{\operatorname{\mathfrak{P}}}
\def\pn{\operatorname{\mathfrak{p}}}
\def\res{\operatorname{res}}
\def\ord{\operatorname{ord}}
\def\AG{\operatorname{AG}}
\setlength{\emergencystretch}{3em}


\begin{titlepage}
{
\centering
\LARGE \textbf{STŘEDOŠKOLSKÁ ODBORNÁ ČINNOST}\\
\Large\textbf{Obor č. 1: Matematika a statistika}\\
\vspace{6cm}
\LARGE\textbf{Medúzy a posloupnosti průměrů}\\
}
\vspace{10cm}
{\noindent\large\bfseries Zdeněk Pezlar\\ 
	\large\bfseries Jihomoravský kraj\\ }
\center\large Brno 2021
	
\end{titlepage}

\begin{titlepage}
{
\centering
\LARGE \textbf{STŘEDOŠKOLSKÁ ODBORNÁ ČINNOST}\\
\Large\textbf{Obor č. 1: Matematika a statistika}\\
\vspace{6cm}
\LARGE\textbf{Medúzy a posloupnosti průměrů}\\
\vspace{1cm}
\LARGE\textbf{On Jellyfish and Sequences of Means}\\
}
\vspace{6cm}
{\noindent\large\bfseries Autor: Zdeněk Pezlar\\ 
	\large\bfseries Škola: Gymnázium Brno, třída Kapitána Jaroše, p. o.\\
    \large\bfseries Kraj: Jihomoravský \\
	\large\bfseries Vedoucí: Mgr. Vojtěch Suchánek\\
	 \large\bfseries Konzultant: Mgr. Marek Sýs Phd.	
	}
	

\end{titlepage}

\newpage
\thispagestyle{empty}
\vspace*{14cm}
\subsubsection*{Prohlášení}

Prohlašuji, že jsem svou práci SOČ vypracoval samostatně a použil jsem pouze prameny a literaturu uvedené v seznamu bibliografických záznamů.
Prohlašuji, že tištěná verze a elektronická verze soutěžní práce SOČ jsou shodné. 
Nemám závažný důvod proti zpřístupňování této práce v souladu se zákonem č. 121/2000 Sb., o právu autorském, o~právech souvisejících s právem autorským a o změně některých zákonů (autorský zákon) v~platném znění. \\[1cm]
V Brně dne: \dotfill \ \ \ \ \ \  Podpis: \dotfill

\newpage
\thispagestyle{empty}
\begin{center}
\includegraphics[width=0.35\textwidth]{podpora_soc-horizontalni.png}
\end{center}
\vspace*{1.5cm}
\begin{center}
\includegraphics[width=0.45\textwidth]{logo_JMK_pruhledne.png}
\end{center}
\vspace*{2.2cm}
\begin{center}
\includegraphics[width=0.35\textwidth]{jcmm-logotype-positive1.png}
\end{center}
\vspace*{6.5cm}
\subsection*{Poděkování}
Tom pozdravuje.


\newpage
\thispagestyle{empty}
\subsection*{Abstrakt}
V naší práci podáme lehký úvod do studia isogenií eliptických křivek bez předchozího studia algebraické geometrie. V práci rovněž diskutujeme několik vybraných protokolů a~poskytujeme úvod do studia algebraické teorie čísel. Pomocí jejího studia pak podrobněji studujeme okruhy endomorfismů supersingulárních křivek. Práce je obohacena o implementace některých zmíněných protokolů, přičemž poskytujeme první implementaci velmi slibného protokolu SITH.


\subsection*{Klíčová slova}
isogenie; eliptická křivka; okruh endomorfismů; grupa tříd ideálů; kvantový počítač; Diffie-Hellman; SIDH; CSIDH; SITH


\vspace*{4cm}

\subsection*{Abstract}
We provide a gentle introduction to the study of elliptic curve isogenies without any assumed knowledge in algebraic geometry. We then discuss several chosen protocols and give a~brief introduction to algebraic number theory. After that, we apply the gained knowledge on the study of endomorphism rings of supersingular curves. The thesis is accompanied by a couple of implemented protocols, providing the first ever implementation of the very promising protocol SITH.

\subsection*{Key words}
isogeny; elliptic curve; endomorphism ring; ideal class group; quantum computer; Diffie-Hellman; SIDH; CSIDH; SITH





{
\hypersetup{linkcolor=black}
\tableofcontents
}
\thispagestyle{empty}

\chapter*{Úvod}
\addcontentsline{toc}{chapter}{Úvod}
\markboth{Úvod}{}

Mějme pro začátek dvě kladná reálná čísla $a,b$. Jejich \textit{aritmetický} a \textit{geometrický průměr} splňují elementární nerovnost:
$$\frac{a+b}{2} \geqslant \sqrt{ab}.$$
Uvažme rekurentní posloupnost dvojic kladných čísel takovou, že každá dvojice je tvořená právě těmito dvěma průměry, tedy $a_0 = a, b_0 = b$ a:
\begin{equation*}
(a_{n+1},b_{n+1}) = \left(\frac{a_n+b_n}{2}, \sqrt{a_n b_n} \right).
\end{equation*}
Platí tedy $a_n \geqslant b_n$ pro kladné $n$. Pro $n$ jdoucí k nekonečnu $a_n$ i $b_n$ konvergují ke společné limitě, tzv. \textit{aritmetickému geometrickému průměru} čísel $a,b$. Tento průměr studoval již Carl Friedrich Gauss [?] a ukázal, že tato na první pohled nevinná posloupnost je spojena s eliptickými integrály. Později se dokonce ukázalo, že tuto posloupnost můžeme využít k rychlému počítání čísla $\pi$ i evaluování funkcí jako $e^x$ či $\arcsin(x)$. 

Co se ale na posloupnost podívat v jiném světle, konkrétně nad konečnými tělesy? V~jistých tělesech můžeme definovat jednoznačně \uv{konzistentní} odmocninu z čísla a tak adaptovat naši posloupnost. Tentokrát posloupnost již ne vždy nekonverguje, zato však tvoří možná zajímavější struktury. Pokud sestavíme orientované grafy popisující naši posloupnost pro všechny dvojice $(a,b)$ nad naším tělesem, získáme grafy, které vypadají následovně:\\
-IMG-\\
Tento graf nazveme \textit{medúzou}. Už to, že grafy tvoří takovéto struktury je pozoruhodné, medúzy ale zde zdaleka nekončí. Ukážeme, že svým způsobem popisují \textit{třídy isomorfismů eliptických křivek} nad naším tělesem, ?. 

Na tomto místě končí článek [?], na kterém je práce založená. My jsme se rozhodli uvážit v potaz i podobné posloupnosti užívající průměry nad konečnými tělesy. Ukážeme, že jedna z nich je s $AG$ posloupnosti téměř shodná, druhá se však liší. Grafy těchto posloupností charakterizujeme a ukážeme, že jsou ještě zajímavější, než pouhé medúzy. Ve finální části práce i tuto posloupnost propojíme s teorií dynamických systémů a eliptických křivek.

\chapter{AG posloupnost nad reálnými čísly}


Nejprve se budeme zabývat posloupností dvojic kladných reálných čísel, přičemž každá další je tvořena aritmetickým a geometrickým průměrem té předchozí. I~v~tomto jednoduchém prostředí narazíme na posloupnost v místech, kde bychom vůbec nehledali.

\section{Seznámení s posloupností}


\begin{definice}
Ať $a,b$ jsou dvě kladná reálná čísla. Pak definujme \textit{AG posloupnost} jako posloupnost $(a_n,b_n)_{n=0}^{\infty}$ tak, že $(a_0,b_0) = (a,b)$ a:
$$\left(a_{n+1},b_{n+1} \right) = \left(\frac{a_n+b_n}{2}, \sqrt{a_n b_n} \right).$$
Jednotlivá čísla $a_i$ a $b_i$ nazveme \textit{složkami} prvku $(a_i,b_i)$ této posloupnosti.
\end{definice}

Toto značení ponechme po zbytek sekce. První vlastnosti, které si všimněme, je monotónnost obou složek $(a_n)_{n=0}^{\infty}$ a $(b_n)_{n=0}^{\infty}$. Z $AG$ nerovnosti je totiž platné $a_n \geqslant b_n$ a~proto:
$$ b_{n+1} = \sqrt{a_n b_n} \geqslant b_n,$$
posloupnost $(b_n)_{n=0}^{\infty}$ je proto rostoucí (pokud $a_0 \neq b_0$, tak ostře rostoucí). Obdobně  můžeme psát:
$$a_{n+1} = \frac{a_n+b_n}{2} \leqslant a_n,$$
posloupnost $(a_n)_{n=0}^{\infty}$ je tedy naopak klesající. Protože průměry dvou čísel leží mezi nimi, obě posloupnosti jsou shora svírané prvkem $a$ a zdola $b$. Libovolná ohraničená monotónní posloupnost konverguje, víme tedy, že obě posloupnosti $(a_n)_{n=0}^{\infty}$ a $(a_n)_{n=0}^{\infty}$ konvergují. Abychom získali nějakou představu o jejich limitách, ukážeme si pár příkladů.


Pokud si zvolíme $a=b=5$, tak jsou obě hodnoty konstantní, to příliš zajímavě není. Zvolme si tedy například trochu záživnější dvojici $a = 2, b = 8$. Pak můžeme psát:


\begin{longtable}[H]{>{\raggedright\arraybackslash}p{0.3\linewidth}p{0.202\linewidth}}
\toprule
$a_i$ & $b_i$\\
\midrule
$2$ & \noindent $8$\\
$5$ & \noindent $4$\\
$ 4.5$ & $4.472135955000\dots$\\
$4.486067977500\dots$ & $4.486046343664\dots$\\
$4.486057160582\dots$ & $4.486057160569\dots$\\ 
$4.486057160575\dots$ & $4.486057160575\dots$\\
$4.486057160575$ & $4.486057160575\dots$\\
$4.486057160575$ & $4.486057160575\dots$\\
\vdots & \vdots\\
\bottomrule 
\end{longtable}
GRAF

V tomto případě prvky $AG$ posloupnosti zdárně konvergují ke společné hodnotě. Spočítejme si ještě pro jistotu jednu posloupnost, tentokrát pro dvojici $a=1$ a $b=\sqrt{2}$. Tuto dvojici vůbec nevolíme náhodně. Vrátíme se k ní ještě za chvíli, její $AG$ posloupnost lze použít k rychlému počítání čísla $\pi$.

\begin{longtable}[H]{>{\raggedright\arraybackslash}p{0.3\linewidth}p{0.202\linewidth}}
\toprule
$a_i$ & $b_i$\\
\midrule
$1$ & $1.414213562373\dots$\\
$1.207106781187\dots$ & $1.189207115003\dots$\\
$1.198156948095\dots$ & $1.198123521493\dots$\\
$1.198140234794\dots$ & $1.198140234677\dots$\\
$1.198140234736\dots$ & $1.198140234736\dots$\\
$1.198140234736\dots$&  $1.198140234736\dots$\\
$1.198140234736\dots$&  $1.198140234736\dots$\\
\vdots & \vdots\\
\bottomrule 
\end{longtable}

Složky $AG$ posloupnosti vždy konvergují ke společné hodnotě.
\begin{veta}\label{conv}
Ať $(a_n,b_n)_{n=0} ^{\infty}$ je AG posloupnost. Pak limity čísel $a_n$ a $b_n$ pro $n$ jdoucí do nekonečna existují a jsou si navzájem rovné. 
\end{veta}
\noindent \textit{Důkaz.} Existenci limit jsme si ukázali výše. Jelikož platí:
$$ 0 \leqslant a_n - b_n  =  2 \left( a_n - \frac{a_n+ b_n}{2} \right) = 2 ( a_{n} - a_{n+1} )$$
a navíc díky existence limity $\lim_{n \rightarrow \infty} a_n - a_{n+1} = 0$, je posloupnost $(a_n-b_n)_{n=0}^{\infty}$ sevřena mezi dvěma posloupnostmi s nulovou limitou, konstantně nulovou posloupností a $(a_{n}-a_{n+1})_{n=0}^{\infty}$, má ji proto nulovou též. \hfill $\square$


\begin{definice}
Tuto společnou limitu nazvěme \textit{aritmeticko-geometrickým průměrem}, zkráceně \textit{AG-průměrem}, čísel $a,b$. Toto číslo značme $\AG(a,b)$.
\end{definice}


Následující věta shrnuje základní vlastnosti $AG$ posloupnosti. 

\begin{veta}\label{zkjb}
Mějme $a,b,k \in \mathbb{R}^{+}$. Pro $AG$ posloupnost platí:
\begin{enumerate}
\item $\AG(a,a) = a$,
\item $\AG(ka,kb) = k \AG (a,b)$,
\item $\AG(a,b) = \AG(a_1,b_1) = \AG(a_2,b_2) = \dots$,
\item $\AG(1-x,1+x) = \AG(a,b)$, kde $x = \frac{1}{a} \sqrt{a^2 - b^2}$.
\end{enumerate}
\end{veta}


Vraťme se zpět k příkladům, které jsme uvedli na začátku. Třetí iterace $AG$ posloupnosti čísel $2$ a $8$ se s limitou shoduje už na čtyřech desetinných místech. Ta následující dokonce na desíti. Opravdu, $AG$ posloupnost konverguje velmi rychle.
\begin{veta}\label{konverg}
Ať $(a_n,b_n)_{n=0}^{\infty}$ je $AG$ posloupnost. Pak složky $(a_n)_{n=0}^{\infty}$ a $(b_n)_{n=0}^{\infty}$ konvergují ke společné limitě kvadraticky.
\end{veta}
Tzv. \textit{řád konvergence} nám říká, jak přesně jak rychle posloupnost konverguje. Přesná definice řádu $\sigma$ posloupnosti $x_i$ konvergující k limitě $L$ je, že pro všechna $n \in \mathbb{N}$ a nějakou konstantu $C$ platí:
$$\frac{\vert x_{n+1} - L \vert}{\vert x_n - L \vert ^ \sigma} \leqslant C.$$
Pro $\sigma = 2$ získáme \textit{kvadraticky konvergentní posloupnost}. U takové posloupnosti se tak v~každém dalším kroku se obě čísla \textit{přibližně} rovnají limitě na dvakrát více desetinných míst.

\noindent \textit{Důkaz.} Zaveďme pomocné posloupnosti $(x_n)_{n=0}^{\infty}$ a $(\varepsilon_i)_{n=0}^{\infty}$ splňující $x_i = \frac{a_i}{b_i} = 1 + \varepsilon_i$ pro každé $i$. Platí $\varepsilon_i \geqslant 0$ pro každé $i$. Pak pro libovolné $n$ platí:
\begin{equation*}
x_{n+1} = \frac{a_n+b_n}{2 \sqrt{a_n b_n}} = \frac{\sqrt{\frac{a_n}{b_n}} + \sqrt{\frac{b_n}{a_n}}}{2} = \frac{\sqrt{x_n}+\frac{1}{\sqrt{x_n}}}{2},
\end{equation*}
takže:
\begin{align*}
1+\varepsilon_{n+1} &= x_{n+1} =\frac{\sqrt{x_n}+\frac{1}{\sqrt{x_n}}}{2}\\
&= \frac{\sqrt{1+\varepsilon_n} + \frac{1}{\sqrt{1+\varepsilon_n}}}{2}.
\end{align*}
Taylorova řada funkce $\sqrt{x}$ v bodě $1$ je $1+\frac{x}{2} - \frac{x^2}{8} + O(x^3)$ a Taylorova řada funkce $\sqrt{x}^{-1}$ je $1-\frac{x}{2}+\frac{3 x^2}{8} + O(x^3)$. Proto pro $n$ dostatečně velké a tedy $\varepsilon_n$ dostatečně malé platí:
$$1+\varepsilon_{n+1}  = 1+\frac{\varepsilon_n ^2}{8} + O(\varepsilon_n ^3),$$ 
řád konvergence $\frac{a_i}{b_i} \rightarrow 1$ je tedy kvadratický. \hfill $\square$\\


Prozatím může vypadat, že tato posloupnost leží na uzavřeném ostrůvku vzdálená od jiných oblastní matematiky. Toto zdání však nemůže být dál od pravdy. Zamysleme se přímo nad samotným průměrem, limity posloupnost. Pro čísla $2$ a $8$ získáváme průměr $4.48605716\dots$. Jak takové číslo určit uzavřeně? K nalezení odpovědi budeme muset nakouknout do sféry tzv. \uv{eliptických integrálů}.
\section{Eliptické integrály}

\begin{definice}
Definujme \textit{eliptický integrál prvního druhu} jako následující určitý integrál:
\begin{equation*}
K(t) := \int_{0}^{\pi/2} \frac{d \theta}{\sqrt{1 - t^2 \sin^2 \theta}}.
\end{equation*}
\end{definice}

Tento integrál a tzv. eliptický integrál \uv{druhé druhu} mají mnoho využití, například v počítání délky oblouku na elipse, ve světě fyziky zase například pomáhají najít periodu kmitu kyvadla [?].

Taktéž jsou intimně spojené s $AG$ posloupností, umožní nám totiž přesně vyjádřit hodnotu $AG(a,b)$.
\begin{veta} (Gauss)
Pro $x<1$ platí:
$$\frac{\pi}{2} \cdot \frac{1}{\AG(1,x)} = K\left(\sqrt{1 - x^2}\right)$$
\end{veta}

Pokud definujeme:
$$I(a,b) := \int_{0}^{\pi/2} \frac{d \theta}{\sqrt{a^2 \sin ^2 \theta + b^2 \cos ^2 \theta}}, $$
tak snadno uvidíme, že $I(a,b) = \frac{1}{a} K(x)$, kde $x =\frac{1}{a} \sqrt{a^2-b^2}$. Takové $x$ jsme už ale někde viděli, konkrétně ve větě \ref{zkjb} iv). Gaussovu větu poté můžeme díky části ii) věty \ref{zkjb} přepsat na:
$$\frac{\pi}{2} \frac{1}{\AG(a,b)} = I(a,b).$$

 Skutečnosti, že se $\frac{1}{AG(1,\sqrt{2})}$ a $\frac{2}{\pi} I(1,\sqrt{2})$ shodují na $11$ místech, si mladý Gauss všiml ve svém deníků již ve svých dvaadvaceti letech. [Pi and AGM] Metoda, kterou pak celou větu dokázal spočívá v důkazu výsledku $I(a,b) = I(a_1,b_1)$, ke kterému dojde po několika přiměřeně bolestných krocích, neboli jak Gauss sám pravil:

\begin{center}
\begin{verse}
\setverselinenums{1}{3}
\textit{\uv{After the development has been made correctly}}
\end{verse}
\end{center}
Důkazem tohoto výsledku jsme hotovi, protože pak v limitním případě $I(a,b) = I(\AG(a,b),\AG(a,b)) = \frac{1}{\AG(a,b)} I(1,1) = \frac{1}{\AG(a,b)} \cdot \frac{\pi}{2}$. Plný důkaz hledejte na [Pi and AGM]. 

Jelikož $AG$ posloupnost konverguje kvadraticky, tato spojitost nám může pomoci počítat právě eliptické integrály velmi rychle. K čemu jinému nám ale rychlá konvergence této posloupnosti může být k užitku?
\section{Rychlé výpočty elementárních funkcí}

$AG$ posloupnost může být využita například při počítání elementárních funkcí. Motivace použití rekurzivně definovaných posloupností může poskytnout například Newtonova metoda pro počítání odmocniny v kvadratickém čase:

\begin{veta}(Newton)
Ať $N>1$ je dané. Pak posloupnost $(x_n)_{n=0}^{\infty}$ splňující $x_0 = N$:
$$x_{n+1} = \frac{1}{2} \left( x_n + \frac{N}{x_n}\right)$$
konverguje kvadraticky k $\sqrt{N}$.
\end{veta}
Důkaz existence a hodnoty limity, ani řádu konvergence není obtížný. Nešlo by obdobně využít i $AG$ posloupnost? Ukáže se, že ano.

Totiž přirozený logaritmus je \uv{přirozeně} spojen s eliptickými integrály [Borw, Pi and AGM]:
$$K(\sqrt{1-x^2}) = \left(1+O(x^2)\right)\ln\left(\frac{4}{x} \right),$$
kde onen chybový člen lze jednoduše odhadnout []. Pro $x$ dostatečně malé nám pak ke spočítání logaritmu velkého čísla postačí použít kvadraticky konvergující $AG$ posloupnost. Dá se jednoduše ukázat, že platí vztahy:
\begin{align*}
\arccos (x) &= \arctan\left(\frac{\sqrt{1-x^2}}{x} \right),\\
\arctan (x) &= \textrm{Im}(\log(1+ix)).
\end{align*}
Pomocí nich a \textit{komplexního AG}, o kterém budeme mluvit dále, pak již můžeme spočítat inverzní funkce k základním goniometrickým funkcím a proto i je samotné.




\section{Posloupnosti s ostatními průměry}

Aritmetický a geometrický průměr nám vygenerovaly posloupnost, která skýtá překvapivě mnoho praktických aplikací. S takovým úspěchem pro jednu dvojici průměrů se pak jenom nabízí vzít v potaz i nějaké další. Zapojíme proto do práce i harmonický průměr, který je pro dvě čísla definován následovně:
$$\frac{2}{\frac{1}{a}+\frac{1}{b}} = \frac{2ab}{a+b}.$$

\begin{definice}
Ať $a,b$ jsou dvě kladná reálná čísla. Pak definujme \textit{HG posloupnost} $(a_n,b_n)_{n=0}^{\infty}$ tak, že $(a_0,b_0) = (a,b)$ a:
$$\left(a_{n+1},b_{n+1} \right) = \left( \frac{2 a_n b_n}{a_n+b_n} , \sqrt{a_n b_n} \right).$$
Obdobně definujeme \textit{AH posloupnost}  $(a_n,b_n)_{n=0}^{\infty}$ tak, že $(a_0,b_0) = (a,b)$ a:
$$\left(a_{n+1},b_{n+1} \right) = \left( \frac{a_n+b_n}{2}, \frac{2 a_n b_n}{a_n+b_n} \right).$$
\end{definice}
Kvůli nerovnostem panujícím mezi průměry můžeme imitovat důkaz věty \ref{conv}, čímž získáme, že obě posloupnosti konvergují k hodnotám $HG(a,b)$, resp. $AH(a,b)$. Abychom tyto posloupnosti porovnali s $AG$ posloupností, spočítejme průměry pro $a=2$ a $b=8$. První posloupnost vypadá následovně:

\begin{longtable}[H]{>{\raggedright\arraybackslash}p{0.3\linewidth}p{0.202\linewidth}}
\toprule
$a_i$ & $b_i$\\
\midrule
$2$ & \noindent $8$\\
$3.2$ & \noindent $4$\\
$3.555555555555\dots$ & $3.577708763999\dots$\\
$3.566597760054\dots$ & $3.566614959874\dots$\\
$3.566606359943\dots$ & $3.566606359954\dots$\\ 
$3.566606359948\dots$ & $3.566606359948\dots$\\
$3.566606359948\dots$ & $3.566606359948\dots$\\
$3.566606359948\dots$ & $3.566606359948\dots$\\
\vdots & \vdots\\
\bottomrule 
\end{longtable} 


\begin{align*}
\left(a_{n+1}, b_{n+1} \right) = \left( \frac{2 a_n b_n}{a_n + b_n}, \sqrt{a_n b_n} \right) = \left( \left(\frac{\frac{1}{a_n}+ \frac{1}{b_n}}{2} \right)^{-1}, \sqrt{a_n ^{-1} b_n ^{-1}} ^{-1}  \right)
\end{align*}

Nyní přichází čas pro $AH$ posloupnost. Bude mít něco společného s předchozími dvěma posloupnostmi? Podívejme se, jak se posloupnost chová s počátečními prvky $a_0 = 2$ a $b_0 = 8$:

\begin{longtable}[H]{>{\raggedright\arraybackslash}p{0.3\linewidth}p{0.202\linewidth}}
\toprule
$a_i$ & $b_i$\\
\midrule
$2$ & \noindent $8$\\
$5$ & \noindent $3.2$\\
$4.1$ & $3.902439024390\dots$\\
$4.001219512195\dots$ & $3.998780859494\dots$\\ 
$4.000000185845\dots$ & $3.999999814155\dots$\\
$4.000000000000\dots$ & $4.000000000000\dots$\\
$4.000000000000\dots$ & $4.000000000000\dots$\\
$4.000000000000\dots$ & $4.000000000000\dots$\\
\vdots & \vdots\\
\bottomrule 
\end{longtable} 

$AH$ posloupnost $2$ a $8$ tedy konverguje zjevně k číslu $4$. Tento úkaz vysvětlí jednoduché pozorování, totiž že součin obou složek je přes všechny prvky posloupnosti konstantní. Platí:
$$a_1 \cdot b_1 = \frac{a+b}{2} \cdot \frac{2ab}{a+b} = ab.$$
Jelikož opět obě složky posloupnosti konvergují ke stejné hodnotě $AH(a,b)$, ta musí splňovat $AH (a,b)^2 = ab$, tedy $AH(a,b) = \sqrt{ab}$. Tento trend, kdy se $AH$ drasticky liší od předchozích dvou, bude v jistém smyslu držet i v pozdějších částech práce, kdy posloupnosti uvažujeme nad konečnými tělesy. Adaptace $AG$ a $GH$ posloupností budou velmi spřízněné, zatímco $AH$ s nimi má velmi málo společného.

Samozřejmě můžeme místo těchto třech průměrů uvažovat libovolné \textit{mocninné průměry} a všechny takové posloupnosti budou konvergovat, to díky platným nerovnostem mezi těmito průměry. Pro mnohem více o teorii s těmito posloupnostmi vřele doporučuji knihu [AG and pi].

Na konec této sekce ještě zmiňme, že se nemusíme zastavit pouze na dvou průměrech. Zobecněná $AGH$ posloupnost pro tři proměnné byla zběžně studovaná v ?, v [?] byly též studovány ještě posloupnosti čtyř a dokonce šesti čísel. Nyní se ale obraťme list a podívejme se více na vlastnosti $AG$ posloupnosti v kontextu teorie čísel. 



\chapter{AG posloupnost nad konečnými tělesy}

Když jsme nyní zodpovědně prozkoumali $AG$ posloupnost nad reálnými čísly, zamysleme se, jaké informace nám $AG$ může poskytnout z pohledu teorie čísel, tedy nad konečnými tělesy. Nepřekvapí nás, že v~konečném případě tato posloupnost skýtá hluboká propojení se zdánlivě nesouvisejícími odvětvími matematiky, konkrétně s \textit{eliptickými křivkami}. O nich ale až později.


\section{Základní poznatky}

Hned ze začátku narážíme na první problém. Ne vždy totiž není součin $a,b \in \mathbb{F}_q$ čtvercem v~$\mathbb{F}_q$ a i pokud je, jak rozlišíme tu správnou odmocninu? Kvůli tomuto problému se prozatím zaměříme na tělesa $\mathbb{F}_q$ s $q = p^k \equiv -1 \pmod{4}$, pak v $\mathbb{F}_q$ neexistuje odmocnina z $-1$. V~každé nenulové dvojici $(x,-x)$ se proto nachází právě jeden čtverec a tak si vždy můžeme zvolit korektní odmocninu, aby byla posloupnost korektně definovaná i dále.  

\begin{poznamka}
Ve skutečnosti jsme na tento problém narazili i nad reálnými čísly, tehdy ale jsou všechna kladná čísla čtverci, tedy je správná volba odmocniny intuitivní.
\end{poznamka}

\begin{definice}
Definujme \uv{zobecněný Legendreho symbol} $\phi_q$ nad $\mathbb{F}_q$ tak, že $\phi_q(0) = 0$ a pro $x$ nenulové je $\phi_q(x)$ rovno $1$, pokud $x$ je v $\mathbb{F}_q$ čtvercem,  a $-1$ jinak.
\end{definice}

\begin{definice}
Ať $a,b$ jsou různé prvky $\mathbb{F}_q ^{\times}$ splňují $\phi_q (ab) = 1$. Pak definujeme $AG_{\mathbb{F}_q}(a,b)$ jako posloupnost $(a_n,b_n)_{n=0}^{\infty}$ s $(a_0,b_0) = (a,b)$ a:
\begin{equation*}
\left(a_{n+1},b_{n+1} \right) = \left(\frac{a_n+b_n}{2}, \sqrt{a_n b_n} \right),
\end{equation*}
přičemž $b_{n+1}$ volíme tak, že $\phi_q (a_{n+1} b_{n+1}) = 1$.
\end{definice}
Všimněme si, že naše posloupnost je dobře definovaná. Aritmetický průměr by nám dělal problém, jen pokud by součet $a_{n+1} + b_{n+1}$ byl nulový. To by znamenalo:
$$a_{n}+b_{n} = - 2\sqrt{a_n b_n}, \qquad \text{takže po umocnění} \qquad (a_n - b_n)^2 = 0.$$

Díky tomu, že odmocniny z čísla jsou navzájem opačná čísla a $\phi_q(-1)=-1$, víme, že pro $a_n b_n \neq 0$ je právě jedno z čísel $\sqrt{a_n b_n}$ a $-\sqrt{a_n b_n}$ čtvercem, mi si $b_{n+1}$ zvolíme tak, aby součin $a_{n+1} b_{n+1}$ byl čtvercem. Můžeme tak pokračovat psát posloupnost i nadále.

Navíc, podmínka $a_i,b_i \in \mathbb{F}_q ^{\times}$ je zachovaná i nadále. Pokud by totiž bylo jedno z čísel $a_{n+1},b_{n+1}$ nulové, jistě to musí být $a_{n+1}$ a tak muselo být $a_{n} = - b_n$, to je ale ve sporu s~volbou $\phi_q (a_n b_n) = 1$. 

Posloupnost budeme vizualizovat jako orientovaný graf, kde hrana vede právě mezi po sobě jdoucími členy posloupnosti.
\begin{definice}
Definujme \textit{roj} (angl. \textit{swarm}) $AG_{\mathbb{F}_q}$ jako orientovaný graf, jehož vrcholy jsou uspořádané dvojice $(a,b)$ prvků nad $\mathbb{F}_q ^{\times}$, jejichž součin je čtverec. Všechny orientované hrany vedou mezi vrcholem $(a,b)$ a $(a_1,b_1)$.
\end{definice}

PRIKLAD, ze meduz. Obrázky!

\begin{veta}\label{pocetprvkuAG}
Graf $AG_{\mathbb{F}_q}$ čítá $(q-1)(q-3)/2$ vrcholů a stejný počet hran.
\end{veta}
\noindent \textit{Důkaz.} Uspořádaná dvojice $(a,b)$ náleží do $AG_{\mathbb{F}_q}$, právě pokud platí $\phi_q(ab) = 1$, tedy buď jsou $a,b$ obě čtverci v $\mathbb{F}_q$, nebo ani jedno. Počet uspořádaných dvojic různých nenulových čtverců je roven $(q-1)/2 \cdot (q-3)/2$ a stejný počet přispívají dvojice nečtverců. Dohromady získáme $2 \cdot (q-1)(q-3)/4$ vyhovujících dvojic. Protože z každého vrcholu vychází právě jedna orientovaná hrana, počet hran je roven počtu vrcholů. \hfill $\square$\\

Grafy z příkladu jsou tvořeny z několika souvislých komponent, které mají všechny velmi specifický tvar, tj. cyklus, kde z každého jeho vrcholu vychází hrana délky jedna. Tento tvar je typický a~libovolná komponenta jej tvoří.

\begin{definice}
Souvislý orientovaný graf $G$ nazveme \textit{medúzou}, pokud je tvořen jediným cyklem $H$ a pro každý vrchol $V \in H$ existuje unikátní předchůdce mimo cyklus, který sám nemá předchůdce.
\end{definice}

Nejprve si charakterizujme, které vrcholy mají v $AG_{\mathbb{F}_q}$ předchůdce, poté již bude popis celého grafu nasnadě.

\begin{lemma}
Vrchol $(a,b) \in AG_{\mathbb{F}_q}$ má předchůdce, právě pokud platí $\phi_q(a^2-b^2)=1$.
\end{lemma}

\noindent \textit{Důkaz.} Nejprve předpokládejme $(a,b)$ má předchůdce $(c,d)$, platí tedy:
\begin{equation*}
a = \frac{c+d}{2}, \quad b = \sqrt{cd}.
\end{equation*}
Potom:
\begin{equation*}
a^2 - b^2 = \left(\frac{c+d}{2} \right)^2 - cd = \left( \frac{c-d}{2} \right)^2
\end{equation*}
je čtverec. Naopak ať $a^2-b^2$ je čtverec a $x$ je nějaká jeho odmocnina. Pak uvažme vrchol $(a-x,a+x)$, jeho následník je: $$\left(\frac{a-x+a+x}{2}, \sqrt{a^2-x^2} \right)= \left(a, b \right),$$
kde $b$ je ta \uv{správná} odmocnina. \hfill $\square$
\begin{veta}\label{meduzy}
Roj $AG_{\mathbb{F}_q}$ je tvořen z několika medúz.
\end{veta}
\noindent \textit{Důkaz}. Graf je určen zobrazením $(u,v) \longmapsto (u_1,v_1) \longmapsto \dots$ na konečné množině, takže každá taková posloupnost jednou vstoupí v cyklus, který bude mít délku větší než $1$. 

Dejme tomu, že $(c,d)$ je členem nějaké cyklu, předchozí člen v cyklu je $(C,D)$, platí $C+D=2c$ a $CD = d^2$, tedy $(C,D)$ jsou kořeny polynomu $x^2-2c+d^2$. Takový polynom má nad $\mathbb{F}_q$ právě dva kořeny, $C$ a $D$. Všichni předkové vrcholu $(c,d)$ v $AG_{\mathbb{F}_q}$ jsou proto $(C,D)$ a $(D,C)$. Díky $q \equiv -1 \pmod{4}$ je $\phi_q(-1)=-1$, proto díky předchozímu lemmatu má právě jeden z těchto dvou vrcholů předchůdce, ten je jistě taky součástí cyklu. Každý vrchol, který není členem cyklus, proto nemá předchůdce a $AG_{\mathbb{F}_q}$ je proto medúzou. \hfill $\square$\\


Pojďme si nyní charakterizovat, jaké různé medúzy můžeme v celém grafu najít. Naše posloupnost je v jistém smyslu homogenní, přesněji podle analogu bodu $ii)$ věty \ref{zkjb}  můžeme přenásobit všechny vrcholy nějakým $k \in \mathbb{F}_q$ a získat \textit{přátelskou} medúzu. Příklady takových medúz jsou na TOM PŘÍKLADU NA ZAČÁTKU.
\begin{definice}
Ať $V \in AG_{\mathbb{F}_q}$ je medúza a $(a,b)$ její prvek. Potom nazveme libovolnou medúzu obsahující prvek $(ka,kb)$ pro $k \in \mathbb{F}_q$ \textit{přítelem} medúzy $V$.
\end{definice}

 Kolik přátel má daná medúza? Na to zodpovídá následující propozice:
\begin{veta}\label{isom}
Ať $(a,b) \in AG_{\mathbb{F}_q}$ leží v cyklu medúzy $M$ a $i$ je první index takový, že existuje $k \in \mathbb{F}_q$ splňující $(a_i,b_i) = (ka, kb)$. Pak má medúza $M$ právě:
$$\frac{q-1}{\ord_q (k)}$$
přátel.
\end{veta}

\noindent \textit{Důkaz.} Je zřejmé, že všechny ostatní prvky cyklu $(a_i,b_i)$ splňující $a_i/b_i = a/b$ jsou ve tvaru $(a_i,b_i) = (k^x a_i, k^x b_i)$. Navíc, pokud přenásobíme všechny prvky $M$ jedním z prvků podgrupy $\mathbb{F}_q \supseteq O_k := \left( \lbrace k, k^2, \dots, k^{\ord_2 (k)} = 1 \rbrace, \times \right)$, pouze otočíme medúzu.

Přesněji, máme danou akci grup $\mathbb{F}_q \times AG_{\mathbb{F}_q} \longrightarrow AG_{\mathbb{F}_q}$, která pro $k \in \mathbb{F}_q$ zobrazí prvek $(a,b)$ na $(ka,kb)$. Nosná množina $O_k$ je pak stabilizátorem pro libovolný prvek medúzy $M$. To znamená, že existuje bijekce mezi množinou prvků $k \in \mathbb{F}_q$, které zobrazí $M$ na medúzu s ní spřátelenou, a faktorgrupou $\mathbb{F}_q/O_k$, která má $\frac{q-1}{\ord_q (k)}$ prvků. \hfill $\square$\\

Pro taxonomické účely se nám hodí tyto spřátelené medúzy uskupit dohromady, zaveďme proto pojem \textit{hejno}.
\begin{definice}
Ať $H \subseteq AG_{\mathbb{F}_q}$ je medúza a $H_1,\dots,H_k$ jsou všichni její přátelé. Pak $H \cup H_1, \cup \dots \cup H_k$ nazvěme \textit{hejnem medúz}.
\end{definice}
 
 
\section{Vlastnosti grafů}

Ohledně medúz je hned několik hodnot, které má cenu zkoumat. Kolik je pro dané $p$ dohromady medúz? Kolik existuje různých hejn? A na jaké délky cyklů můžeme narazit? Pojďme se na tyto hodnoty podívat trochu podrobněji.

Asi nejdůležitější hodnotou je pro nás počet medúz v celém hejnu. Tuto hodnotu studovali autoři původního článku [?] a pomocí eliptických křivek budeme moci uvést odhady na tato čísla.

\begin{definice}
Ať $q \equiv 3 \pmod{4}$ je mocnina prvočísla. Pak označme $d(\mathbb{F}_q)$ počet všech medúz v grafu $AG_{\mathbb{F}_q}$. Navíc, označme $s(\mathbb{F}_q)$ počet všech hejn v grafu $AG_{\mathbb{F}_q}$.
\end{definice}
V článku, ze kterého vycházíme, se $d(\mathbb{F}_q)$ nazývá \textit{jellyfish number}, číslo $s(\mathbb{F}_q)$ není zmíněno vůbec a obecně hejna medúz nejsou nijak značena a jsou zmíněna pouze okrajově. Protože víme z příkladu ?, že délky cyklů se přes prvočísla mohou tak lišit, tak nás nepřekvapí, že i celkový počet medúz se chová poměrně různorodě. Pro představu uveďme malou tabulku pro prvočísla $p < 100$.

\begin{longtable}[H]{>{\raggedright\arraybackslash}p{0.15\linewidth}p{0.15\linewidth}}
\toprule
$p$ & $d(\mathbb{F}_p)$\\
\midrule
$3$ & \noindent $9$\\
$7$ & \noindent $1$\\
$11$ & \noindent $3$\\
$19$ & \noindent $8$\\
$23$ & \noindent $5$\\
$31$ & \noindent $10$\\
$43$ & \noindent $7$\\
$47$ & \noindent $4$\\
$59$ & \noindent $7$\\
$67$ & \noindent $30$\\
$71$ & \noindent $25$\\
$79$ & \noindent $18$\\
$83$ & \noindent $6$\\
\bottomrule 
\end{longtable}

Tato náhodná povaha se nese i dál, na následujícím grafu vidíme jednotlivé hodnoty pro prvočísla $p < 10^5$:

Graf.

I po sobě jdoucí prvočísla mohou mít disproporcionálně různé počty medúz. Na příklad pro prvočíslo $1619$ je počet medúz roven $d(\mathbb{F}_{1619}) = 56$ a hned o dům dál u prvočísla $1627$ nalezneme v grafu enormní počet $d(\mathbb{F}_{1627}) =2227$ medúz, skoro čtyřicetkrát více.

Autoři původního článku ukázali, že pro libovolně malé $\varepsilon > 0$ a $q$ dostatečně velké platí:
$$d(\mathbb{F}_q) \geqslant \left( \frac{1}{2} - \varepsilon \right) \sqrt{q},$$
pomocí teorie eliptických křivek, o kterých se budeme bavit v pozdější části práce. V závěru práce spekulují, jak optimální tento odhad je a navrhují

\section{HG posloupnost}


V první kapitole jsme si ukázali, že pokud místo aritmetického a geometrického průměru zvolíme jinou dvojici průměrů, získáme posloupnosti úzce propojené s $AG$ posloupností. Co tedy se podívat na jejich obdoby v konečných tělesech? 


Nejprve zapojme do práce geometrický a harmonický průměr, kde samozřejmě definujeme harmonický průměr dvou nenulových čísel s nenulovým součtem jako:
$$\frac{2}{\frac{1}{a}+\frac{1}{b}} = \frac{2ab}{a+b},$$
kde $\frac{1}{a}$ je multiplikativní inverze čísla $a$. Pilný čtenář si ověří, že s multiplikativními inverzemi můžeme pracovat obdobně jako v reálných číslech. Definujme pak $HG$-posloupnost nad konečným tělesem.


\begin{definice}
Ať $a,b$ jsou různé prvky $\mathbb{F}_q ^{\times}$ splňují $\phi_q (ab) = 1$. Pak definujeme $HG_{\mathbb{F}_q}(a,b)$ jako posloupnost $(a_n,b_n)_{n=0}^{\infty}$ s $(a_0,b_0) = (a,b)$ a:
\begin{equation*}
\left(a_{n+1},b_{n+1} \right) = \left(\frac{2}{\frac{1}{a_n} + \frac{1}{b_n}}, \sqrt{a_n b_n} \right),
\end{equation*}
přičemž $b_{n+1}$ volíme tak, že $\phi_q (a_{n+1} b_{n+1}) = 1$.
\end{definice}

\begin{definice}
Definujme \textit{roj} $HG_{\mathbb{F}_q}$ jako orientovaný graf, jehož vrcholy jsou uspořádané dvojice $(a,b)$ prvků nad $\mathbb{F}_q ^{\times}$, jejichž součin je čtverec. Všechny orientované hrany vedou mezi vrcholem $(a,b)$ a $(a_1,b_1)$.
\end{definice}
PRIKLADY, GRAF.


Při porovnání předchozího příkladu se můžeme dovtípit, že tato posloupnost je pouze přestrojená AG posloupnost. V tomto přesvědčení nás může utvrdit počet hran a vrcholů i kritérium, kdy vrchol má předchůdce.

\begin{veta}
Graf $HG_{\mathbb{F}_q}$ čítá $(q-1)(q-3)/2$ vrcholů a stejný počet hran.
\end{veta}
\textit{Důkaz.} Analogický k důkazu věty \ref{pocetprvkuAG}. \hfill $\square$


\begin{lemma}
Vrchol $(a,b) \in HG_{\mathbb{F}_q}$ má předchůdce, právě pokud platí $\phi_q(b^2-a^2)=1$.
\end{lemma}

\noindent \textit{Důkaz.} Nejprve předpokládejme $(a,b)$ má předchůdce $(c,d)$, platí tedy:
\begin{equation*}
a = \frac{2cd}{c+d}, \quad b = \sqrt{cd}.
\end{equation*}
Potom:
\begin{equation*}
b^2-a^2 = cd- \left(\frac{2 cd}{c+d} \right)^2 = cd \left( \frac{c-d}{c+d} \right)^2
\end{equation*}
je čtverec, protože pracujeme pouze s dvojicemi, jejichž součin je čtvercem. Naopak ať $b^2-a^2$ je čtverec a $x$ je nějaká jeho odmocnina. Pak uvažme vrchol $\Big(\frac{b^2+b  x}{a},\frac{b^2-bx}{a} \Big)$, jeho následník je: $$\left(\frac{2 b^2 (b+x)(b-x)}{a^2 \left(\frac{b^2+bx}{a} + \frac{b^2-bx}{a} \right)}, \sqrt{\frac{b^2(b^2-x^2)}{a^2}} \right)= \left(\frac{2 b^2 \cdot a^2}{2 a \cdot b^2 }, b \right) = \left(a, b \right),$$
kde $b$ vybíráme jako tu \uv{správnou} odmocninu z $b^2$. \hfill $\square$

\begin{dusledek}
Graf $GH_{\mathbb{F}_q}$ je tvořen z několika medúz.
\end{dusledek}
\noindent \textit{Důkaz}. Analogický k důkazu věty \ref{meduzy}. \hfill $\square$\\

Rozdíl mezi oběma grafy je ten, že vrchol $(a,b)$ pro $a,b$ je součástí cyklu v \textit{právě jednom} z grafů $AG_{\mathbb{F}_q}$ a $HG_{\mathbb{F}_q}$. To nám napovídá, jaké bude konkrétní propojení těchto dvou grafů.

\begin{veta}
Platí isomorfismus grafů $AG_{\mathbb{F}_q} \cong HG_{\mathbb{F}_q}$.
\end{veta}
\noindent \textit{Důkaz.} Uvažme zobrazení $\psi : AG_{\mathbb{F}_q} \longrightarrow HG_{\mathbb{F}_q}$ určené předpisem $\psi((a,b)) = (1/a,1/b)$. Ukážeme, že toto zobrazení definuje mezi grafy isomorfismus. Opravdu, uvažme orientovanou hranu v grafu $AG_{\mathbb{F}_q}$: 
$$(a,b) \longmapsto \left(\frac{a+b}{2}, \sqrt{ab} \right),$$
poté v grafu $HG_{\mathbb{F}_q}$ má $\phi_q(a,b)$ hranu:
$$\psi (a,b) = \left(\frac{1}{a}, \frac{1}{b} \right) \longmapsto \left(\frac{2/ab}{1/a+1/b}, \sqrt{\frac{1}{ab}} \right) = \left( \frac{2}{a+b}, \frac{1}{\sqrt{ab}} \right) = \psi \left( \frac{a+b}{2}, \sqrt{ab} \right).$$
\hfill $\square$\\


\chapter{AH posloupnost}
Zatím jsme pracovali s dvěma dvojicemi průměrů z trojice - aritmetický, geometrický a~harmonický. Co se proto podívat i na tu poslední? Tentokrát již ze začátku nebude pracovat pouze nad konečným tělesem, ale i s bodem v nekonečnu. Přesto se můžeme ptát, jak souvisí tato posloupnost a její grafy s předchozími dvěma, obzvláště ve spojení s eliptickými křivkami.

\begin{definice}
Ať $K$ je těleso. Pak definujeme $\mathbb{P}^{1} (K)$ jako $K \cap \lbrace \infty_i  \vert i \in \mathbb{F}_q ^{\times} \rbrace$, kde $\infty$ je \textit{bod v nekonečnu}. Platí $\frac{1}{0} = \infty$, $\frac{1}{\infty} = 0$, $m+\infty = \infty$ a konečně $\infty_m \times 0 = m$ pro $m \in K$.
\end{definice}

Poněkud zvláštní definice několika nekonečen nám bude vhod později.

K této ani $HG$ posloupnosti nad konečnými tělesy neexistuje podle nejlepšího svědomí autora žádná literatura. Strávíme nějaký čas nad tvary grafů - případ $AH$ posloupnosti je totiž na dvakrát tolik zajímavý, jako ty předchozí.

\section{Základní poznatky}


\begin{definice}
Ať $a,b \in \mathbb{F}_q ^{\times}$ jsou různé. Pak definujeme $AH_{\mathbb{F}_q}(a,b)$ jako posloupnost $(a_n,b_n)_{n=0}^{\infty}$ s $(a_0,b_0) = (a,b)$ a:
\begin{equation*}
\left(a_{n+1},b_{n+1} \right) = \left(\frac{a_n+b_n}{2}, \frac{2}{\frac{1}{a_n} + \frac{1}{b_n}} \right),
\end{equation*}
pokud $a+b \neq 0$. Prvek $(a,-a)$ pro $a \in \mathbb{F}_q$ se zobrazí na $(0,\infty_k)$, $(0,\infty_k)$ se zobrazí na $(\infty_k,0)$ a $(\infty_k,0)$ se zobrazí sám na sebe. Zde $k = -a^2$.
\end{definice}

Pokud $\phi_q(2) \neq 1$, tak se každý \textit{afinní} prvek zobrazí opět na afinní prvek, tedy body, jejich některá složka je $0$, tvoří vlastí komponentu. Ať je naopak pro nějaká $(a_0,b_0)$ a $n$ nezáporné $1/a_{n+1} + 1/b_{n+1} = 0$, pak i $a_{n+1} + b_{n+1} = 0$. Muselo pak být:
\begin{align*}
\frac{a_n+b_n}{2} + \frac{2 a_n b_n}{a_n + b_n} &= 0,\\
(a_n+b_n)^2 + 4 a_n b_n &= 0,\\
\left(\frac{a_n}{b_n} + 1 \right)^2 + 4 \frac{a_n}{b_n} &= 0,\\
\left(\frac{a_n}{b_n}\right)^2 + 6 \frac{a_n}{b_n} + 1 &= 0.
\end{align*}
Poznamenejme, že $b_n \neq 0$. Tato kvadratická rovnice má kořen nad $\mathbb{F}_q$, právě pokud $2$ je v~$\mathbb{F}_q$ čtvercem. Pro tělesa, kde $2$ je čtvercem, se některé prvky mohou zobrazit do nekonečna. My se nejprve zaměříme na tělesa $\mathbb{F}_q$ s $q \equiv \pm 3 \pmod{8}$.

\begin{definice}
Definujme \textit{graf} $AH_{\mathbb{F}_q}$ jako orientovaný graf, jehož vrcholy jsou uspořádané dvojice $(a,b)$ prvků nad $\mathbb{P}^1 (\mathbb{F}_q ^{\times})$. Všechny orientované hrany vedou mezi vrcholem $(a,b)$ a $(a_1,b_1)$.
\end{definice}

PRIKLAD

Na příkladu vidíme, že pro $q \equiv \pm 3 \pmod{8}$ při vizualizaci $AH$ posloupnosti získáme krom medúz i tzv. \textit{vulkány hloubky $2$}. V případě $q \equiv \pm 1 \pmod{8}$ jsou tyto vulkány dokonce ještě hlubší a některé obsahují $(\infty,0)$. Tato terminologie není vybraná autorem, setkáme se s ní v kontextu eliptických křivek.
\begin{definice}
Souvislý orientovaný graf $V$ nazveme \textit{vulkánem hloubky $k$}, pokud je dělen do $k+1$ stupňů $V_0,\dots,V_k$ a:
\begin{enumerate}
\item $V_0$ je (možná triviální) cyklus, kde každý jeho člen má unikátního následníka mimo cyklus,
\item pro $0 <i < k$ má každý vrchol $W \in V_i$ unikátního předchůdce ve $V_{i-1}$ a dva následníky ve $V_{i+1}$,
\item každý prvek $V_k$ je listem.
\end{enumerate}
\end{definice}
Všimněme si, že medúza je pouze vulkánem hloubky $1$. Předtím, než ukážeme, že grafy $AH$ posloupnosti pro $q = \pm 3 \pmod{8}$ nabývají těchto tvarů, se pozastavme nad spojením $AH$ posloupnosti s dvěma předchozími, které jsme studovali. I když větší vulkány $AG$ posloupnost nikdy netvoří, pro například $p \equiv -1 \pmod{4}$ získáme v některých případech $AH$ posloupnosti též medúzy. Klíčové rozdělení bude na prvky $(a,b)$, kde $\phi_q (ab)$ je fixní. Hned uvidíme, že toto číslo je pro jednotlivé souvislé komponenty stejné a dokážeme silnější tvrzení. Určeme nyní počty (afinních) dvojic v jednotlivých takových skupinách.

\begin{veta}
Buď $q = p^k$ mocnina prvočísla. Pak:
\begin{enumerate}
\item počet prvků $(a,b) \in AH_{\mathbb{F}_q}$ takových, že $\phi_q(ab) = 1$, je $(q-1)(q-3)/2$, 
\item počet prvků $(a,b) \in AH_{\mathbb{F}_q}$ takových, že $\phi_q(ab) = -1$, je $(q-1)^2/2$.
\end{enumerate}
Počet hran v celém grafu vycházejících z afinních vrcholů je $(q-1)(q-2)$.
\end{veta}  
\noindent \textit{Důkaz.} V případě, kdy $ab$ je v $\mathbb{F}_q$ nenulovým čtvercem, leží v roji $AH_{\mathbb{F}_q}$ právě dvojice $(a,b)$, až na případ, kdy $a=b$. Pokud je součin dvou prvků čtverec, tak jsou buď oba čtverce, nebo ani jeden. Počet dvojic nenulových prvků $(a,b)$, jejichž součin je čtverec, spočítáme tedy součtem počtů dvojic různých čtverců, resp. nečtverců. Toto je $(q-1)/2 \cdot (q-3)/2 + (q-1)/2 \cdot (q-3)/2 =  (q-1)(q-3)/2$.

V případě, kdy $ab$ není v $\mathbb{F}_q$ nenulovým čtvercem, leží v roji $AH_{\mathbb{F}_q}$ právě dvojice $(a,b)$. Takové dvojice mají jedno složku, která je čtvercem, a druhou, která není. Vyhovující počet je proto $(q-1)/2 \cdot (q-1)/2 + (q-1)/2 \cdot (q-1)/2 = (q-1)^2/2$. Konečně, z každého afinního vrcholu vychází právě jedna hrana, proto počet hran je:
$$\frac{(q-1)(q-3)}{2}+\frac{(q-1)^2}{2} = (q-1)(q-2).$$  \hfill $\square$\\



Grafy $AH_{\mathbb{F}_q}$ a $AG_{\mathbb{F}_q}$ jsou velmi odlišné. Na příkladu ? vidíme, že komponenty roje $AG_{\mathbb{F}_q}$ mohou mít mnohonásobně více prvků, než je $q$. Zato v případě $AH$ posloupnosti počet prvků značně omezí stejný invariant, jako v reálném případě - součin jednotlivých složek prvků. Zde poprvé využijeme naši definice nekonečna.

\begin{lemma}\label{fix}
Uvažme graf $AH_{\mathbb{F}_q}$ a nějakou jeho souvislou komponentu $V$. Pro libovolnou afinní dvojici $(a,b) \in V$ je $ab$ fixní.
\end{lemma}
\noindent \textit{Důkaz.} Stačí nám ukázat, že pro vrchol $(a,b)$ a jeho následníka platí $a_1 b_1 = ab$, jelikož $(a_1,b_1)$ má právě dva předchůdce, $(a,b)$ a $(b,a)$. A opravdu:
\begin{equation*}
a_1 \cdot b_1 = \frac{a+b}{2} \cdot \frac{2ab}{a+b} = ab.
\end{equation*} 
Toto platí i v případě, že jedním z prvků $a_1,b_1$ je nekonečno.
\hfill $\square$\\

\begin{dusledek}\label{fixab}
Každá souvislá komponenta v grafu $AH_{\mathbb{F}_q}$ obsahuje nejvýše $q-1$ afinních vrcholů.
\end{dusledek}
\noindent \textit{Důkaz.} Pro dané $k \in \mathbb{F}_q$ je nad $\mathbb{F}_q$ jistě $q-1$ dvojic se součinem $k$, konkrétně $\left(a, \frac{k}{a}\right)$ pro $a \in \mathbb{F}_q ^{\times}$. Podle předchozího lemmatu mají všechny prvky jedné souvislé komponenty stejný součin prvků, je jich proto nejvýše $q-1$. \hfill $\square$\\

Poznamenejme, že z těchto $q-1$ prvků ne nutně všechny vyhovují, v tělese $\mathbb{F}_{11}$ a součin roven čtyřem nevyhovuje dvojice $(2,2)$. Lemma, které jsme zmínili před chvílí, nám též umožní zobecnit větu \ref{isom}, tentokrát je totiž počet přátel grafů k dané souvislé komponentě velmi omezený.
\begin{definice}
Ať $(a,b) \in AG_{\mathbb{F}_q}$ leží v souvislé komponentě $V$. Potom nazveme libovolnou souvislou komponentu obsahující prvek $(ka,kb)$ pro $k \in \mathbb{F}_q$ přítelem $V$.
\end{definice}

\begin{veta}
Ať $(a,b) \in AG_{\mathbb{F}_q}$ leží v souvislé komponentě $V$, která obsahuje pouze afinní vrcholy. Pak počet přátel $V$ je roven:
\begin{enumerate}
\item $q-1$, pokud $(-a,-b)$ neleží ve $V$,
\item $(q-1)/2$, pokud $(-a,-b)$ leží ve $V$.
\end{enumerate} 
\end{veta}
\noindent \textit{Důkaz.} Důkaz je prakticky stejný, jako důkaz věty \ref{isom}, tentokrát ale pokud pro $k \neq 1$ leží $(a,b)$ a $(ka,kb)$ ve stejné komponentě, podle lemmatu \ref{fix} musí platit $a b = k^2 ab$, tj. $k = - 1$. Nosná množina grupy $O_{k}$ sestrojené analogicky k důkazu věty \ref{isom} je proto podmnožinou $\lbrace 1,-1 \rbrace$ a dojdeme k tomu, že  $V$ má právě $\frac{q-1}{\ord_q (\pm 1)} \in  \left\lbrace q-1,\frac{q-1}{2}\right\rbrace$ přátel. \hfill $\square$\\

V případě, že komponenta obsahuje body v nekonečnu, pak předchůdci prvku $(0,\infty_m)$ jsou $(\pm a,\mp a)$ pro $a^2 = -m$, tedy tato komponenta má $(q-1)/2$ přátel. Poznamenejme, že ve zdánlivé většině grafů $AH_{\mathbb{F}_q}$ se vyskytují komponenty s $q-1$ přáteli, stejně jako jiné komponenty, které mají přátel pouze $(q-1)/2$

\begin{definice}
Ať $H \subseteq AH_{\mathbb{F}_q}$ je souvislá komponenta roje a $H_1,\dots,H_k$ jsou všichni její přátelé. Pak $H \cup H_1 \cup \dots \cup H_k$ nazvěme \textit{hejnem}.
\end{definice}
 

\section{Struktura grafů}

$AH$ posloupnost se od $AG$ posloupnosti na několika místech principiálně liší, přesto se na jednom místě shodují. Jejich grafy mají pozoruhodně pravidelnou strukturu. V pozdějších částech práce tuto strukturu do jisté míry vysvětlíme. Bez dalšího otálení proto pojďme opravdu dokázat, že grafy $AH_{\mathbb{F}_q}$ mají tu strukturu, kterou jim připisujeme. Nejprve samozřejmě klasifikujeme, kdy má prvek předchůdce.

\begin{lemma}
Afinní vrchol $(a,b) \in AH_{\mathbb{F}_q}$ má předchůdce, právě pokud $\phi_q(a^2-ab)=1$.
\end{lemma}

\noindent \textit{Důkaz.} Nejprve předpokládejme $(a,b)$ má předchůdce $(c,d)$, platí tedy:
\begin{equation*}
a = \frac{c+d}{2}, \quad b = \frac{2cd}{c+d}.
\end{equation*}
Potom:
\begin{equation*}
a(a-b) = \frac{c+d}{2} \left( \frac{c+d}{2} - \frac{2cd}{c+d} \right) =\left( \frac{c-d}{2}\right)^2
\end{equation*}
je čtverec. Naopak ať $a(a-b)$ je čtverec a $x$ je nějaká jeho odmocnina. Pak uvažme vrchol $(a-x,a+x)$, jeho následník je: $$\left(\frac{a-x+a+x}{2}, \frac{2 (a-x)(a+x)}{a-x+a+x} \right)= \left(a, b \right).$$ \hfill $\square$\\

Toto tvrzení je velmi zajímavé tím, jak nám rozdělí práci pro případy, kdy $q$ dává zbytek $3$ a $5$ po dělení osmi. Pokud se totiž podíváme na čísla, která udávají, kdy vrcholy $(a,b)$ a~$(b,a)$ mají předchůdce - $a(a-b)$ a $b(b-a)$ - tak jejich součin je:
$$-ab (a-b)^2.$$  
V případě $q \equiv 3 \pmod{8}$ není $-1$ v $\mathbb{F}_q$ čtvercem a proto $a,b$ s $\phi_q (ab) = 1$ má právě jeden z vrcholů $(a,b)$, $(b,a)$ předchůdce. Pokud $ab$ není čtverec, tak buď oba vrcholy mají předchůdce, nebo ani jeden. Samozřejmě pro $q \equiv 5$ je tato situace prohozena. Toto rozdělení nám též pomůže odhalit, proč některé grafy jsou medúzy a jiné jsou vulkány větší hloubky.
\begin{veta}\label{big}
Roj $AH_{\mathbb{F}_q}$ vypadá následovně:
\begin{enumerate}
\item Pokud $q \equiv 3 \pmod{8}$, tak:
\begin{itemize}
\item sjednocení komponent obsahujících prvky $(a,b)$ splňující $\phi_q(ab) = 1$ je tvořeno medúzami,
\item sjednocení komponent obsahujících prvky $(a,b)$ splňující $\phi_q(ab) = -1$ je tvořeno vulkány hloubky $2$.
\end{itemize}
\item Pokud $q \equiv 5 \pmod{8}$, tak:
\begin{itemize}
\item sjednocení komponent obsahujících prvky $(a,b)$ splňující $\phi_q(ab) = 1$ je tvořeno vulkány hloubky $2$,
\item sjednocení komponent obsahujících prvky $(a,b)$ splňující $\phi_q(ab) = -1$ je tvořeno medúzami.
\end{itemize}
\end{enumerate}
Vrcholy obsahující body v nekonečnu tvoří v $AH_{\mathbb{F}_q}$ samostatné komponenty.
\end{veta}

\noindent \textit{Důkaz.} Uvažujme bez újmy na obecnosti $q \equiv 3 \pmod{8}$, tedy $\phi_q(-1)= -1$, případ $q \equiv 5 \pmod{8}$ je analogicky, pouze se prohodí role vrcholů $(a,b)$ s $\phi_q(ab) =1$ a $\phi_q(ab) = -1$. Nejprve zmiňme, že díky $\phi_q(2) = -1$ leží vrcholy s body v nekonečnu v $q-1$ samostatných komponentách. Ukážeme, že vrcholy $(a,b)$, pro které není součin $ab$ čtvercem, tvoří medúzy, a pokud $\phi_q (ab) = 1$, tak tvoří vulkány hloubky $2$. 

K tomu užijeme právě fakt, jenž jsme naznačili výše, že pokud $ab$ je čtverec, tak právě jeden z vrcholů $(a,b)$ a $(b,a)$ má předchůdce. Jako v případě $AG$ posloupnosti tedy vyberme libovolný vrchol $(a,b)$, $a \neq b$, a hledejme další členy posloupnosti $(a_1,b_1), (a_2,b_2)$, \dots, dokud nedojdeme do cyklu. Ať $(c,d)$ je členem cyklu a jeho předchůdci jsou vrcholy $(C,D)$, $(D,C)$. Víme, že jeden z těchto dvou nemá předchůdce a ten druhý proto musí být členem cyklu. Graf je tedy medúzou.

Nyní přijde ta zajímavější část, tedy že pokud $ab$ čtvercem není, tak náš graf je tvořen vulkány. Uvažme vrchol $V : (a,b)$ grafu, který má následníka, bez újmy na obecnosti uvažme $a=1$. Podle lemmatu je $1-b$ čtverec v $\mathbb{F}_q$, tj. $b = 1-x^2$ pro nějaké $x \in \mathbb{F}_q$. Nyní spočítejme dva následníky $V$:
$$ (1,b) \longmapsto \left(\frac{b+1}{2}, \frac{2b}{b+1} \right) \longmapsto \left( \frac{b^2+6b+1}{4(b+1)}, \frac{4b(b+1)}{b^2+6b+1} \right) =: W. $$

Jádro celé charakterizace spočívá v následujícím tvrzení:

\begin{lemma*}
Vrchol $W$ má na druhé větvi předchůdců nejvýše prarodiče.
\end{lemma*}
\noindent \textit{Důkaz.}  Opravdu, druhý předchůdce $W$ je jistě:
$$Z : \left(\frac{2b}{b+1}, \frac{b+1}{2} \right).$$
ten má sám předchůdce, protože číslo:
$$\frac{2b}{b+1} \left( \frac{2b}{b+1} -  \frac{b+1}{2} \right) = \frac{-b (b-1)^2}{(b+1)^2}$$
je čtvercem. Ať $(X,Y)$ a $(Y,X)$ jsou předchůdci $Z$, ti pak splňují soustavu:
\begin{align*}
\frac{X+Y}{2} &= \frac{2b}{b+1},\\
\frac{2XY}{X+Y} =  \frac{b+1}{2} \Rightarrow XY &= b.
\end{align*}
Jsou tedy kořeny kvadratické rovnice $U^2 - \frac{4b}{b+1} U + b = 0$ nad $\mathbb{F}_q$.  Kořeny této rovnice spočítáme a získáme:
$$\lbrace X,Y \rbrace = \left\lbrace \frac{2b + \sqrt{-b}(b-1)}{b+1},\frac{2b - \sqrt{-b}(b-1)}{b+1} \right\rbrace.$$

Poznamenejme, že ty leží v $\mathbb{F}_q$, protože $\phi_q(-b) = \phi_q(-1)\cdot \phi_q(b) = (-1)^2 = 1$. Konečně ukážeme, že $(X,Y)$ nemá předchůdce, z toho jistě plyne, že i $(Y,X)$ nemá předchůdce. K~tomu nám stačí ověřit, že číslo:
\begin{align*}
\frac{2b + \sqrt{-b}(b-1)}{b+1} \cdot \left( \frac{2b + \sqrt{-b}(b-1)}{b+1} - \frac{2b - \sqrt{-b}(b-1)}{b+1} \right) &=\\
\frac{2b + \sqrt{-b}(b-1)}{(b+1)^2} \cdot 2 \sqrt{-b}(b-1)&=\\
\frac{2 (b-1)}{(b+1)^2} \cdot [2 \sqrt{-b} b - b(b-1)] &=\\
\frac{2 b(b-1)}{(b+1)^2} \cdot [2 \sqrt{-b}-(b-1)] &= \frac{2 b(b-1)}{(b+1)^2} (1-\sqrt{-b})^2.
\end{align*}
není v $\mathbb{F}_q$ čtvercem. Protože víme, že $\phi_q(2) = -1$, tak $2b$ je v $\mathbb{F}_q$ druhá mocnina. Číslo $b-1 = -x^2$ naopak čtverec není, protože $q \equiv -1 \pmod{4}$. Levý činitel proto čtvercem není. Protože $b$ není čtverec a $q \equiv 3 \pmod{4}$, tak argument v druhé mocnina $1-\sqrt{-b}$ leží v $\mathbb{F}_q$, tudíž pravý činitel je druhou mocninou čísla v $\mathbb{F}_q$. Dohromady získáme, že $\phi_q (X(X-Y)) = -1$, čímž je pomocné tvrzení dokázáno. \hfill $\square$

\begin{poznamka}
Pokud bychom uvažovali tělesa, kde $\phi_q(2) =1$, potom by měl vrchol $W$ z~obou stran alespoň praprarodiče.
\end{poznamka}

Co jsme právě udělali? Ukázali jsme, že pokud nějaký vrchol, v tomto případě je to $W$, má na jedné větvi předků alespoň praprarodiče, tak na té druhé má už nejvýše prarodiče. Nyní již se můžeme pustit přímo do důkazu, že naše posloupnost tvoří vulkány.

Stejně jako v případě $AG$ posloupnosti začněme s libovolným vrcholem $(a,b)$. Pišme jeho následníky:
$$(a,b)\longmapsto (a_1,b_1)\longmapsto (a_2, b_2) \longmapsto \dots$$
Máme nekonečně definovanou posloupnost na konečné množině vrcholů, jednou proto vstoupí do cyklu, který má délku větší než jedna. Každému prvku v tomto cyklu (i kdyby byl cyklus kratší, než $4$) můžeme psát nekonečně dlouho, byť periodickou, řadu předchůdců. Dejme proto tomu, že $(C,D)$ je libovolným členem cyklu ve stejné partitě, ve které leží $(a,b)$. Právě jsme dokázali, že na větvi předchůdců, která nezasahuje do cyklu, má právě tři předchůdce, jednoho rodiče a dva prarodiče. Toto platí pro libovolný člen cyklu. Tímto tedy získáváme, že každá partita v tomto případě tvoří vulkány hloubky $2$.

 \hfill $\square$\\
 
Tato charakterizace byla poměrně pracná, přesto je pouze polovina války vyhrána. Zaprvé, co když uvážíme konečná tělesa $\mathbb{F}_q$, kde $q = p^k$ a sice platí $p \equiv 3,5 \pmod{8}$, ale $k$ je sudé? V takovém případě je $\phi_q(2)=1$ a v případě vulkánů má každý bývalý list nového předchůdce. Na příklad rozšiřme příklad ? nad tělesem $\mathbb{F}_{p^2}$, pak graf vypadá následovně:

Vulkán má tedy o jedna vyšší hloubky. V případě rozšíření lichého stupně jsou grafy shodné, při rozšíření sudého stupně můžeme získat alespoň jednoduché odhady na hloubku binárního stromu, který je připojen ke členu cyklu. Důkaz, že všechny listy mají stejnou hloubku přes všechny takové stromy, tedy že graf je opět vulkánem, je již nad možnosti základní teorie čísel. 

\begin{dusledek}
Buď $q = p^m$ a $V \subseteq AH_{\mathbb{F}_q}$ vulkán hloubky $h$ a $(a,b)$ nějaký jeho prvek. V grafu $AH_{\mathbb{F}_{q^k}}$ leží $(a,b)$ ve stromu zakořeněném v cyklu. Potom výška tohoto stromu je alespoň $h+v_2(k)$.
\end{dusledek}
\noindent \textit{Důkaz.} Postupujme indukcí podle $v_2(k)$. Případ $k$ lichého pokrývá věta \ref{big}. Ať nyní věta platí pro nějaké $\ell \geqslant 0$ a všechna $k$ s $v_2 (k) = \ell$. Pokud $(a,b)$ je list v $\mathbb{F}_{q^k}$ pro nějaké $k$, pak platí $\phi_{q^k} (a(a-b)) = -1$ a tedy $a(a-b)$ je čtvercem v $\mathbb{F}_{p^{2k}}$. Vrchol $(a,b)$ má proto dva předchůdce $(a\pm x,a\mp x) \in AH_{\mathbb{F}_{q^{2k}}}$  a výšla stromu obsahujícího $(a,b)$ má v $AH_{\mathbb{F}_{q^{2k}}}$ hloubku alespoň o jedna delší, než v $AH_{\mathbb{F}_{q^{k}}}$. Snadno pak získáme dokazované tvrzení. \hfill $\square$\\

Vzpomeňme si na známé lemma z olympiádní matematiky, konkrétně \textit{Lifting the Exponent lemma}, které hodnotu $v_2(k)$ ukotví k číslu $p^k - 1$. 
\begin{veta}(LTE lemma)
Ať $p$ je liché a $k$ sudé. Pak platí:
$$v_2 (p^k - 1) = v_2(p-1)+v_2 (p+1) + v_2 (k) - 1.$$
\end{veta}
Důsledek výše spolu s větou \ref{big} pak ukazuje, že hloubka vulkánu je určitým způsobem spojena s $v_2 (q-1)$. Toto propojení plně prozkoumáme až ke konci práce i pro tělesa s~charakteristikou $p \equiv \pm 1 \pmod{8}$ pomocí eliptických křivek.

\section{Vlastnosti grafů}

I v případě $AH$ posloupnosti se můžeme dívat na empirická data ohledně jednotlivých parametrů.


\section{Dynamické systémy}

AH posloupnost se od dvou, které jsme studovali před chvílí, liší také tím, že nemusíme nijak svévolně vybírat tu \uv{správnou} odmocninu. Tato posloupnost je tím mnohem jednodušeji studovatelná, protože je udaná zobrazeními, která jsou pouze lomenými funkcemi. 
  
V AH posloupnosti zobrazíme prvek $(x,1)$ na $\left(\frac{x+1}{2}, \frac{2x}{x+1}\right)$. Jaké poznatky můžeme vytěžit, kdybychom i tento prvek znovu normalizovali na $\left ( \frac{(x+1)^2}{4x}, 1 \right)$? Poté se zabýváme iterací zobrazení:
$$x \longmapsto \frac{(x+1)^2}{4x}$$
a jejím chováním na $\mathbb{F}_q$. Toto je přesně úkolem oblasti matematiky studující \textit{dynamické systémy} lomených funkcí nad konečnými tělesy.

Dynamické systémy byly přes poslední dekády hojně zkoumány, i přesto se o nich ví poměrně málo. Přehledový článek z roku 2013 [?] dává do kontextu, kolik jejich struktury je nám zatím neznámo, dokonce i pouhé očekávané chování dynamického systému. 




Obecný kecy.

Pohled dynamických systémů nám pomůže najít největší cyklus, na který můžeme narazit.

--říct, že cyklus v AH je ekvivalentně s cyklem Dynamic system--

Rovnice: $$\frac{(x+1)^2}{4x} = a$$ s parametrem $a$ má nad $\mathbb{F}_q$ řešení, právě pokud diskriminant výsledné kvadratické rovnice $x^2 + (2-4a)x+1=0$, tedy $4(1-a)^2-4 = 4a(a-1)$, je nad $\mathbb{F}_q$ čtvercem.

???

%\begin{definice}
%Ať $\chi : \mathbb{F}_q ^{\times} \longrightarrow \mathbb{F}_q ^{\times}$ je zobrazení %splňující $\chi (a) \chi(b) = \chi(ab)$ pro $a,b \in \mathbb{F}_q ^{\times}$. Pak $\chi$ %nazveme \textit{multiplikativním charakterem} na $\mathbb{F}_q$.
%\end{definice}

%Typické příklady multiplikativních charakterů na $\mathbb{F}_q$ jsou triviální charakter %$\varepsilon$, které každý prvek zobrazí na $1$, či Legendreho symbol $\phi_q$. %Multiplikativní charaktery se originálně studovaly kvůli hledání řešení rovnic jako %například $x^3 + y^3 = 1$ nad konečnými tělesy, pomocí Jacobiho sum můžeme ale obecněji %najít počet řešení rovnice typu $a_1 x_1 ^{b_1} + \dots + a_k x_k ^{b_k} = c$. 

%\begin{definice}
%Ať $\chi, \lambda$ jsou multiplikativní charaktery na $\mathbb{F}_q$. Pak %\textit{Jacobiho sumu} $J(\chi,\lambda)$ definujeme jako:
%$$J (\chi, \lambda) = \sum_{a+b = 1} \chi (a) \lambda (b).$$
%\end{definice}

\begin{veta}
Počet $a \in \mathbb{F}_q \setminus \lbrace ? \rbrace$ takových, že $\phi_q ( a^2-a) = 1$ je $\frac{q-(-1)^{(q+1)/4}}{4}$.
\end{veta}
\noindent \textit{Důkaz.} Určíme součet:
$$\sum_{a} \phi_q (a (a-1)) = \sum_{a} \phi_q (a) \phi_q (a-1).$$
Protože pro libovolné $a \neq 1$ platí $\phi_q (a-1) ^2 = 1$, tak díky multiplikativitě $\phi_q$ máme $\phi_q (a-1) = \phi_q (a-1)^{-1}$. Tím získáme:
\begin{align*}
\sum_{a} \phi_q (a) \phi_q (a-1) &=  \sum_{a} \phi_q (a) \phi_q (a-1)^{-1}\\
&= \sum_{a} \phi_q \left (\frac{a}{a-1} \right).
\end{align*}
Nyní přejdeme na proměnnou $x = \frac{a}{a-1}$, které splňuje $a = \frac{x}{x-1}$. Pro každé $x \not\in \lbrace \rbrace$ existuje $a \not\in \lbrace \rbrace$, takže blabla ok.

Jelikož Legendreho symbol nabývá pouze hodnot $0,1$ a $-1$, přičemž v našem součtu figuruje $?$ sčítanců, z nich dva nulové. To znamená, že právě ? z nich je rovno jedné, což jsme chtěli. \hfill $\square$\\

\begin{poznamka}
Tento součet a ostatně i trik, kde podělíme výraz $\phi_q (a-1)^2$, je inspirován teorií obklopující tzv. \textit{Jacobiho sumy} multiplikativních charakterů nad $\mathbb{F}_q$. Konkrétně součet $\sum \phi_q (a(a-1))$ je roven $\phi_q (-1) \sum \phi_q (a) \phi_q (1-a)$, což je Jacobiho suma $\phi_q (-1) J(\phi_q, \phi_q)$. Tyto sumy jsou intimně spojené s počtem řešení rovnic typu $a_1 x_1 ^{b_1} + \dots + a_n x_n ^{b_n} = k$ nad konečnými tělesy. Pro excelentní, byť mírně pokročilý, úvod do jejich studia doporučuji [IR]. 
\end{poznamka}

\begin{dusledek}
Každý cyklus, který udává dynamika $\frac{(x+1)^2}{4x}$, má délku nejvýše $\frac{q-1}{4}$.
\end{dusledek}

\noindent \textit{Důkaz.}
\hfill $\square$


Jak moc je tento odhad těsný? Pokud si spočítáme poměr $p$ a největšího cyklu, který nalezneme nad $\mathbb{F}_p$, tak získáme následující data:
-obrázek-

Vidíme tedy, že poměrně mnoho prvočísel dosahuje této maximální délky cyklu, další trendy se poté drží u podílů $1/5, 1/6$ a poté většina prvočísel spadá pod podíl $1/12$. (vysvětlení?)


\chapter{Propojení s eliptickými křivkami}

Je pozoruhodné, že tak jednoduchá věc, jako $AG$ či $HG$ posloupnost, generuje nad konečnými tělesy tak pravidelné grafy jako medúzy. Toto není vůbec náhoda, podobné grafy totiž popisují mnohem složitější struktury, konkrétně grafy isogenií eliptických křivek nad konečnými tělesy.

\section{Rychlý úvod do eliptických křivek}
V této sekci rychle a svižně probereme základy teorie eliptických křivek nad konečnými tělesy. Pro podrobnější text nemohu nedoporučit svou SOČ [], další excelentní cizojazyčné zdroje jsou [],[],[].

Po celou dobu se pohybujeme v tzv. \textit{projektivní prostoru}, tedy množině tříd nenulových vektorů $(a_0: \dots: a_n) \in \overline{K}^{n+1}$, kde dva vektory považujeme za shodné, pokud jsou vzájemně skalárními násobky. Tyto třídy nazveme \textit{body}.


\begin{definice}
Ať $A,B,\lambda \in \mathbb{F}_q$ jsou taková, že $4 A^3 + 27 B^2 \neq 0$ a $\lambda \neq 0, 1.$ Pak definujeme \textit{eliptickou křivku ve Weierstrassově tvaru} jako množinu bodů $(x,y) \in \mathbb{F}_q$ splňujících vztah:
$$y^2 = x^3 + Ax + B,$$
spolu s tzv. \textit{bodem v nekonečnu} $O$. Dále definujeme \textit{eliptickou křivku v Legendrově tvaru} jako množinu $(x,y) \in \mathbb{F}_q$ splňujících:
$$y^2 = x(x-1)(x-\lambda),$$
opět s bodem v nekonečnu.
\end{definice}
Pokud definujeme sčítání na křivce tak, že součet každých tří kolineárních (ne-nutně různých) bodů je $O$, pak body na eliptické křivky tvoří grupu. V případě, že přímka $PQ$ pro $P,Q$ body na $E$ degeneruje v tečnu, pak poslední průsečík této přímky s $E$ bude dvojnásobek bodu $P$. Díky asociativitě sčítání na křivky můžeme pak jednoznačně definovat $n$-násobek bodu $[n]P = \underbrace{P+\dots+P}_{n}$.

--Obrázek--


\begin{figure}[h]
\begin{center}
\begin{tikzpicture}
        \begin{axis}[
            xmin=-2,
            xmax=4,
            ymin=-3,
            ymax=3,
            scale only axis,
            axis lines=middle,
            % set the minimum value to the minimum x value
            % which in this case is $-\sqrt[3]{7}$
            domain=-1.3247:2.1663,      % <-- works for pdfLaTeX and LuaLaTeX
%            domain=-1.91293118:3,   % <-- would also work for LuaLaTeX
            samples=200,
            smooth,
            % to avoid that the "plot node" is clipped (partially)
            clip=false,
            % use same unit vectors on the axis
        ]
            \addplot [red] {sqrt(x^3-x+1)};
            \addplot [red] {-sqrt(x^3-x+1)};
        \end{axis}
        \draw[black!30!green] (4.78,5.485) -- (0.975,4);
         \fill[blue] (0.975,4) circle (1.25 pt);
           \node[left, outer sep=2pt] at (0.975,4) {$P$};
           \fill[blue] (2.975,4.78) circle (1.25 pt);
           \node[above right, outer sep=2pt] at (2.975,4.78) {$Q$};
          \fill[blue] (4.78,5.485) circle (1.25 pt);
           \node[right, outer sep=2pt] at (4.78,5.485) {$-(P+Q)$};
             \fill[blue] (4.78,1.785) circle (1.25 pt);
           \node[right, outer sep=2pt] at (4.78,1.785) {$P+Q$};
           \draw[dashed] (4.78,5.485) -- (4.78,1.785);
             
         \draw[->] (-0.5,3.63) -- (-0.5,6);
         \draw[->] (-0.5,3.63) -- (-0.5,1.35);
         \node[above right, outer sep=2pt] at (-0.5,6) {$\mathcal{O}$};
        \end{tikzpicture}
\end{center}
\caption{Doslova ten samej obrázek !!!!!!!}
\end{figure}


Grupa bodů definovaných nad konečným tělesem je isomorfní direktnímu součinu $\mathbb{Z}_{n} \times \mathbb{Z}_m$ pro vhodná celá $m,n$ [já]. Pokud označíme $E(\mathbb{F}_q)$  množinu bodů na $E$ definovaných nad $\mathbb{F}_q$ (včteně $O$), tak zmiňme důležitou \textit{Hasseho větu}, která značně ukotví počet prvků této křivky. Ta tvrdí, že:
\begin{equation*}
\vert q+1 - \# E(\mathbb{F}_q) \vert \leqslant 2\sqrt{q}.
\end{equation*}

Důležité pro nás jsou zobrazení mezi křivkami, která zachovají jejich grupovou strukturu.

\begin{definice}
Ať $E_1,E_2$ jsou eliptické křivky nad tělesem $K$. Surjektivní homomorfismus grup $\phi: E_1(\overline{K}) \longrightarrow E_2(\overline{K})$ tvaru $\phi : (x:y:z) \longmapsto (u(x,y,z):v(x,y,z):w(x,y,z))$ pro polynomy $u,v,w \in K[x]$ nazveme \textit{isogenií}. Pod jádrem $\ker \phi$ isogenie $\phi$ rozumíme jejímu jádru jako homomorfismu grup. 
\end{definice}

Obzvláště důležité isogenie jsou \textit{isomorfismy}, tzn. invertibilní isogenie - isogenie dané lineárními zobrazeními $(x,y) \longmapsto (ax+by+c,dx+ey+f)$. Je jednoduché ukázat, že pro křivky ve Weierstrassově tvaru jsou isomorfismy dané zobrazením $(x,y) \longmapsto (u^2 x, u^3 y)$ pro $u \in \overline{K}$. Pokud je isomorfismus $\phi : E \longrightarrow E^{\prime}$ definovaný nad rozšířením $K$, ale ne přímo nad $K$, pak řekneme, že je $E^{\prime}$ \textit{twistem} křivky $E$. Je-li takový isomorfismus definovaný nad rozšířením $K(\sqrt{u})$, pak jej nazveme \textit{kvadratickým twistem}.

I když ne všechny isogenie jsou invertibilní, ke každé isogenii $\phi : E \longrightarrow E^{\prime}$ najdeme její \textit{duální isogenii} $\widehat{\phi} : E^{\prime} \longrightarrow E$. Můžeme proto říci, že \uv{být isogenní} je relace ekvivalence na množině křivek nad daným tělesem. Jak ale zjistit, kdy jsou dvě křivky isogenní? Částečný výsledek nám může poskytnout věta připisovaná \textit{Sato a Tatovi}:
\begin{veta}\label{satotate} (Sato-Tate)
Buďte $E,E^\prime$ eliptické křivky nad $\mathbb{F}_q$. Pak jsou tyto křivky isogenní nad $\mathbb{F}_q$, právě pokud platí $\#E (\mathbb{F}_q) = \#E^\prime (\mathbb{F}_q)$.
\end{veta}

Problém, kdy jsou dvě křivky isogenní pod isogenií daného stupně, je již obtížnější.

\section{Okruhy endomorfismů}

\section{AG posloupnost ve světle eliptických křivek}

\chapter{Eliptické křivky a AH posloupnost}

V procesu studia AH posloupnosti jsem se pokusil propojit tuto posloupnost s teorií obklopující eliptické křivky, podobně jako autoři původního článku ? udělali s AG posloupností. Přímo adaptovat jejich postup, tedy přiřadit vrcholům grafu křivky, podle nejlepšího mínění autora není pravděpodobné, protože komponenty $AH_{\mathbb{F}_q}$ mají velmi jednoduché invarianty a mají velmi málo prvků. To by znamenalo, že bychom museli vybírat ideály s malými řády v grupě tříd ideálů. 

Ne, $AH$ posloupnost můžeme popsat trochu jednodušeji. Přesto se ale vrátíme do světa eliptických křivek. Víceméně.

\section{Motivace}

Pro tuto sekci proto sledujme trochu pozorněji časovou osu studia dynamických systémů, mnohé z nich vedly směrem isogenií přímo na křivkách, tzv. \textit{endomorfismů}.

Vraťme se hned čtyři dekády nazpět, kdy Miller a Koblitz stáli u zrodu kryptografie pomocí eliptických křivek. Pro strukturovanější úvod do studia šifrování pomocí eliptických křivek a konkrétněji isogenií opět skromně doporučuji konzultovat mou práci []. Při studiu šifrování ryze nad eliptickými křivkami, tedy se zabýváme pouze skalárními isogeniemi $[n]P$, hledáme křivky, které obzvláště elegantní vzorce pro násobení, zpravidla hledáme jednoduché vzorce pro $[2]P$ a $[3]P$. Právě Neal Koblitz zjistil, že pro eliptickou křivku (ne ve Weiestrassově tvaru):
$$E : y^2 + xy = x^3 + 1$$
nad konečným tělesem charakteristiky $2$ mají body velmi pěkné dvojnásobky:
$$?$$ 
Mohou nám však pomoci studovat i dynamiku funkce $x + \frac{1}{x}$ nad tělesy $\mathbb{F}_{2^n}$. V [?] je totiž ukázáno, že pro bod $P = (x,y) \in E$ a $\pi : (x,y) \longmapsto (x^2,y^2)$ Frobeniův automorfismus platí:
$$P + \pi (P) = \left(x + \frac{1}{x} , \quad x^2 + y + 1 + \frac{1}{x^2} + \frac{y}{x^2} \right).$$
Pokud se tedy zaobíráme čistě $x$-ovou souřadnicí, endomorfismus $1+\pi$ zobrazí $x$ na prvek $x+\frac{1}{x}$.


\section{Montgomeryho křivky}

Ve snaze adaptovat postup popsaný výše, jsem hledal křivky, na nichž existuje endomorfismus $\phi$ zobrazující bod $P : (x,y)$ na bod s $x$-ovou složkou $\frac{(x+1)^2}{4x}$. U křivek ve Weierstrassově tvaru se mi takové zobrazení najít nepodařilo. Připomeňme, že i v popisu $AG$ posloupnosti přece pracujeme s jiným tvarem křivek, není tedy příliš překvapivé, že to bude třeba i nyní.

Hledaný endomorfismus jsem nakonec nalezl u křivek v \textit{Montgomeryho tvaru} $ B y^2 = x^3 + A x^2 +x$ pro $A,B \in \mathbb{F}_q$. Volíme zde $A \neq \pm 2$, v těchto případech je totiž křivka singulární a má mnohem jednodušší strukturu. Tyto křivky mají několik praktických výhod, proto se například používají v šifrovacím protokolu CSIDH, pro více informací doporučuji konzultovat mou předchozí práci [já].

První z takových výhod je, že třída isomorfismů Montgomeryho křivky závisí \textit{pouze} na hodnotě parametru $A$. Další výhodou je, že lomené funkce udávající zobrazení $[2]$ a $[3]$ mají na Montgomeryho křivkách jednoduší tvary, což umožňuje pro rychlejší výpočty. Konkrétně pro bod $P = (x,y)$ je jeho dvojnásobek roven:
$$[2] P = \left(\frac{(x^2-1)^2}{4x(x^2+Ax+1)},y \frac{(x^2-1)(x^4+2 A x^3 + 6x^2 + 2 Ax + 1)}{8 x^2 (x^2+Ax+1)^2} \right).$$
viz [karaskova]. Pokud bychom zvolili $A = -2$, tak $x$-ová souřadnice bodu $[2]P$ by byla právě $\frac{(x+1)^2}{4x}$, což hledáme. Toto pozorování otevírá cestu studiu $AH$ posloupnosti pomocí eliptických křivek.

Problém ale nastává právě s hodnotou $A=-2$, pro ni je totiž křivka rovna $y^2 = x(x-1)^2$. Tato křivka ve skutečnosti není eliptická, v bodě $(1,0)$ je singulární a nelze v něm spočítat tečnu. Všechny ostatní body při klasicky definovaném sčítání opět tvoří grupu, tentokrát je ale opravdu jednodušší, než na klasické eliptické křivce.

\begin{definice}
Uvažme křivku $E : y^2 = x(x-1)^2$ nad tělesem $\mathbb{F}_q$. Definujeme $E(\mathbb{F}_q)$ jako grupu bodů $(x,y) \in \mathbb{F}_q ^2$ splňující $y^2 = x(x-1)^2$ a $x \neq 1$ spolu s bodem v nekonečnu $\mathcal{O}$. 
\end{definice}



ma to $p+1$ bodů, isomorfni Z $p+1$. 




\begin{center}
\begin{verse}
\setverselinenums{1}{3}
\textit{It is possible to write endlessly on elliptic curves (This is not a threat.)}
\end{verse}
\hfill \textit{Serge Lang}
\end{center}


\chapter*{Závěr}
\addcontentsline{toc}{chapter}{Závěr}
\markboth{Závěr}{}
zu ende

Carl Friedrich Gauss aritmeticko-geometrický průměr ve svém mládí studoval hojně, v jeho deníku o této posloupnosti nalezneme hned destíku zmínek této posloupnosti mezi roky $1799$ a $1800$. Věnoval se i zobecnění posloupnosti nad komplexními čísly. Jak to zobecnit nad konečnýma tělesama - - p adický?





%\begin{algorithm}
%\caption{A}
%\begin{algorithmic}
%\REQUIRE $n \geq 0 \vee x \neq 0$
%\ENSURE $y = x^n$
%\STATE $y \leftarrow 1$
%\IF{$n < 0$}
%\STATE $X \leftarrow 1 / x$
%\STATE $N \leftarrow -n$
%\ELSE
%\STATE $X \leftarrow x$
%\STATE $N \leftarrow n$
%\ENDIF
%\WHILE{$N \neq 0$}
%\IF{$N$ is even}
%\STATE $X \leftarrow X \times X$
%\STATE $N \leftarrow N / 2$
%\ELSE[$N$ is odd]
%\STATE $y \leftarrow y \times X$
%\STATE $N \leftarrow N - 1$
%\ENDIF
%\ENDWHILE
%\end{algorithmic}
%\end{algorithm}


\chapter*{Použitá značení}
\begin{flalign*}
&a \mid b  &&a \text{ dělí } b\\
&\frac{1}{a}  &&\text{multiplikativní inverze } a,\text{ tj. } a^{-1} \\
&\nu_p(n) &&p\text{-adická valuace } n\\
&\genfrac{(}{)}{}{}{a}{p} && \text{Legendreův symbol } a \text{ vzhledem k } p\\
&\mathbb{N},\mathbb{Z},\mathbb{Q},\mathbb{R},\mathbb{C} &&\text{množina přirozených, celých, racionálních, reálných, komplexních čísel} \\
&\mathbb{Z}_d &&\text{okruh zbytků modulo } d \\
&\mathbb{F}_q &&\text{konečné těleso s } q \text{ prvky}\\
&\overline{K} &&\text{algebraický uzávěr tělesa } K\\
&K^{\times} &&\text{multiplikativní podrupa tělesa } K\\
&\mathbb{P}^{n}(K) &&\text{projektivní prostor nad } K \text{ o dimenzi } n+1\\
&E(K) &&\text{množina bodů křivky } E \text{ nad } K\\
&\# E(K) &&\text{počet bodů na křivce } E \text{ nad konečným tělesem } K\\
&\mathcal{O},O &&\text{bod v nekonečnu křivky } E\\
&[n]_E,[n] &&\text{násobení } n \text{ na křivce } E\\
&\pi,\pi_E &&\text{Frobeniův endomorfismus}\\
&\widehat{\phi} &&\text{isogenie duální k } \phi\\
&\deg \phi &&\text{stupeň isogenie } \phi\\
&\ker \phi &&\text{jádro isogenie } \phi\\
&\# \ker \phi &&\text{velikost jádra isogenie } \phi\\
&\langle G\rangle &&\text{podgrupa generovaná množinou } G\\
&E/G &&\text{obraz } E \text{ v separabilní isogenii s jádrem } G\\
&E/\mathfrak{a} &&\text{obraz } E \text{ v isogenii generované ideálem } \mathfrak{a}\\
&E[n] &&n\text{-torze křivky } E\\
&\End(E) &&\text{okruh endomorfismů } E\\
&\mathrm{Ell}_{\mathcal{O}} &&\text{množina eliptických křivek nad } \mathbb{F}_p \text{ s okruhem endomorfismů } \End(E) \cong \mathcal{O}\\
& M \otimes_{R} N &&\text{tenzorový součin } R\text{-modulů } M \text{ a } N\\
&\End ^0(E) &&\text{algebra endomorfismů } E\\
&\Tr \phi, \Tr \alpha && \text{stopa endomorfismu } \phi \text{, stopa } \alpha \in \End^0 (E)\\
&\N \alpha && \text{norma } \alpha \in \End^0 (E)\\
&\widehat{\alpha} && \text{Rosatiho involuce } \alpha \in \End^0 (E)\\
&j(E) &&j\text{-invariant křivky } E\\
&G_{\ell} (\overline{\mathbb{F}}_p) &&\text{graf supersingulárních } j \text{-invariantů nad } \overline{\mathbb{F}}_p \text{ spojených isogeniemi stupně } \ell\\
&R[x] &&\text{okruh polynomů s koeficienty nad okruhem } R\\
&K(a_1,\dots, a_n) &&\text{nejmenší nadtěleso } K \text{ obsahující prvky } a_1, \dots, a_n\\  
&[K:L] &&\text{stupeň rozšíření tělesa } K \text{ nad } L\\
&\alpha(x) &&\text{lineární transformace } x \mapsto \alpha x \text{ působící na } \mathbb{Q}(\theta)\\ 
&M_{\alpha} &&\text{matice odpovídající } \alpha(x)\\ 
&\Tr M &&\text{stopa matice } M\\ 
&\det M &&\text{determinant matice } M\\
&Tr_K(\alpha) &&\text{stopa prvku } \alpha \text{ v } K\\ 
&N_K(\alpha) &&\text{norma prvku } \alpha \text{ v } K\\ 
&\mathcal{O}_K &&\text{okruh celých algebraických čísel tělesa } K\\
&Cl(\mathcal{O}) &&\text{grupa tříd ideálů pořádku } \mathcal{O}\\
&h_{\mathcal{O}} &&\text{řád grupy } Cl(\mathcal{O})\\
&(a) &&\text{hlavní ideál generovaný prvkem } a\\
&\frac{\mathfrak{a}}{m} &&\text{lomený ideál } \frac{\mathfrak{a}}{m}\\
&N_{\mathcal{O}}(\mathfrak{a}) &&\text{norma ideálu } \mathfrak{a} \subseteq \mathcal{O}, \text{ tj. } \vert \mathcal{O}/\mathfrak{a} \vert\\
&\mathfrak{a} + \mathfrak{b} &&\text{součet ideálů } \mathfrak{a} \text{ a } \mathfrak{b}\\
&\mathfrak{a} \mathfrak{b}, \mathfrak{a} \cdot \mathfrak{b} &&\text{součin ideálů } \mathfrak{a} \text{ a } \mathfrak{b}\\
&\mathfrak{a} \vert \mathfrak{b} &&\text{ideál } \mathfrak{a} \text{ dělí ideál } \mathfrak{b}\\
&\mathsf{G} / \mathsf{H} &&\text{faktorgrupa } \mathsf{G} \text{ podle } \mathsf{H}\\
&\deg f&&\text{stupeň polynomu, lomené funkce } f\\
&f^{\prime}&&\text{derivace } f\\
&f \vert_{M} && \text{zúžení } f \text{ na množinu } M\\
&\phi \vert_{\ell} && \text{zúžení isogenie } \phi \text{ na } \ell\text{-torzi}\\
&f \in O(g) &&f \text{ roste asymptoticky nejvýše stejně rychle jako } g
\end{flalign*}


\begin{thebibliography}{97}

\bibitem{SIKE}
\textsc{Azarderakhsh}, Reza, Matthew \textsc{Campagna}, Craig \textsc{Costello}, Luca \textsc{De Feo}, Basil \textsc{Hess}, Amir \textsc{Jalali}, Brian \textsc{Koziel}, Brian \textsc{LaMacchia}, Patrick \textsc{Longa}, Michael \textsc{Naehrig}, Joost \textsc{Renes}, Vladimir \textsc{Soukharev} a David \textsc{Urbanik}: \textit{SIKE: Supersingular Isogeny Key Encapsulation}. 2017.

\bibitem{CSIFISH}
\textsc{Beullens}, Ward, Thorsten \textsc{Kleinjung} a Frederik \textsc{Vercauteren}: \textit{CSI-FiSh: Efficient Isogeny based Signatures through Class Group Computations}. 2019. Dostupné z: \url{https://eprint.iacr.org/2019/498}.

\bibitem{Dark}
\textsc{Bottinelli}, Paul, Victoria \textsc{de Quehen}, Christopher \textsc{Leonardi}, Anton \textsc{Mosunov}, Filip \textsc{Pawlega} a Milap \textsc{Sheth}: \textit{The Dark SIDH of Isogenies}. ISARA Corporation, Waterloo, Canada. 2019. Dostupné z: \url{https://eprint.iacr.org/2019/1333}.


\bibitem{Bisson}
\textsc{Bisson}, Gaetan a Andrew V. \textsc{Sutherland}: \textit{Computing the Endomorphism Ring of an Ordinary Elliptic Curve Over a Finite Field}. 2009. Dostupné z: \url{https://arxiv.org/abs/0902.4670}.

\bibitem{CSIDH}
\textsc{Castirik}, Wouter, Tanja \textsc{Lange}, Chloe \textsc{Martindale}, Lorenz \textsc{Panny} a Joost \textsc{Renes}: \textit{CSIDH: An Efficient Post-Quantum Commutative Group Action.} 2018.

\bibitem{Prase}
\textsc{Čermák}, Filip a Matěj \textsc{Doležálek}: \textit{Teorie nejen čísel}. Seriál korespondenčního matematického semináře.

\bibitem{eSIDH}
\textsc{Cervantes-Vázquez}, Daniel, Eduaro \textsc{Ochoa-Jiménez} a Francisco \textsc{Rodríguez-Henríquez}: \textit{eSIDH: the revenge of the SIDH}. 2020.

\bibitem{Chen}
\textsc{Chen}, Evan: \textit{An Infinitely Large Napkin}. Dostupné z: \url{https://venhance.github.io/napkin/Napkin.pdf}.

\bibitem{Childs}
\textsc{Childs}, Andrew, David \textsc{Jao} a Vladimir \textsc{Soukharev}: \textit{Constructing elliptic curve isogenies in quantum subexponential time}. Journal of Mathematical Cryptology,8(1), 2014. Dostupné z: \url{https://arxiv.org/abs/1012.4019}

\bibitem{Chuang}
\textsc{Chuang}, Isaac L. a Michael A. \textsc{Nielsen}: \textit{Quantum Computation and Quantum Information}. Cambridge University Press, Cambridge, 2000. 

\bibitem{Conrad1}
\textsc{Conrad}, Keith: \textit{Trace and Norm}. University of Connecticut, Connecticut. Dostupné z: \url{https://kconrad.math.uconn.edu/blurbs/galoistheory/tracenorm.pdf}.

\bibitem{Conrad2}
\textsc{Conrad}, Keith: \textit{Ideal Factorization}. University of Connecticut, Connecticut. Dostupné z: \url{https://kconrad.math.uconn.edu/blurbs/gradnumthy/idealfactor.pdf}.

\bibitem{Conrad3}
\textsc{Conrad}, Keith: \textit{The Conductor Ideal}. University of Connecticut, Connecticut. Dostupné z: \url{https://kconrad.math.uconn.edu/blurbs/gradnumthy/idealfactor.pdf}.

\bibitem{BSIDH} 
\textsc{Costello}, Craig: \textit{B-SIDH: supersingular isogeny Diffie-Hellman using twisted torsion}. Microsoft Research, USA, 2019. Dostupné z: \url{https://eprint.iacr.org/2019/1145}.

\bibitem{Costello}
\textsc{Costello}, Craig: \textit{Supersingular isogeny key exchange for beginners}. Microsoft Research, USA, 2019. Dostupné z: \url{https://eprint.iacr.org/2019/1321}.


\bibitem{Couveignes}
\textsc{Couveignes}, Jean-Marc: \textit{Hard Homogenous Spaces}. 2006. Dostupné z: \url{https://eprint.iacr.org/2006/291.pdf}.

\bibitem{Cox}
\textsc{Cox}, David: \textit{Primes of the form $x^2+n y^2$ : Fermat, Class Field Theory and Complex Multiplication}. New York, 1989.

\bibitem{DeFeo}
\textsc{De Feo}, Luca: \textit{Fast Algorithms for Towers of Finite Fields and Isogenies}. EcolePolytechnique X, 2010.

\bibitem{DeFeo3}
\textsc{De Feo}, Luca, David \textsc{Jao} a Jérôme \textsc{Plût}: \textit{Towards quantum-resistant cryptosystems from supersingular elliptic curve isogenies}. Math. Cryptol. 8(3): 209-247, 2014. Dostupné z: \url{https://eprint.iacr.org/2011/506.pdf}.

\bibitem{DeFeo2}
\textsc{De Feo}, Luca: \textit{Mathematics of Isogeny Based Cryptography}. Université de Versailles \& Inria Saclay, 2017. Dostupné z: \url{https://arxiv.org/abs/1711.04062}.

\bibitem{DeFeo4}
\textsc{De Feo}, Luca: \textit{Isogeny based Cryptography: what’s under the hood?} École des Mines de Saint-Étienne, Gardanne, 2018. Dostupné z: \url{http://defeo.lu/docet/talk/2018/11/15/gardanne/}.

\bibitem{DeFeo5}
\textsc{De Feo}, Luca, Jean \textsc{Kieffer} a Benjamin \textsc{Smith}: \textit{Towards practical key exchange from ordinary isogeny graphs}. 2018. Dstupné z: \url{https://eprint.iacr.org/2018/485}.

\bibitem{SeaSign}
\textsc{De Feo}, Luca a Steven \textsc{Galbraith}: \textit{SeaSign: Compact isogeny signatures from class group actions}. EUROCRYPT 2019. Dostupné z: \url{https://eprint.iacr.org/2018/824}.

\bibitem{SQISign}
\textsc{De Feo}, Luca, David \textsc{Kohel}, Antonin \textsc{Leroux}, Christopher \textsc{Petit} a Benjamin \textsc{Wesolowski}: \textit{SQISign: compact post-quantum signatures from quaternions and isogenies}. 2020. Dostupné z: \url{https://eprint.iacr.org/2020/1240}.

\bibitem{Science}
\textsc{Deng}, Yu-Hao, Xing \textsc{Ding}, Lin \textsc{Gan}, Peng \textsc{Hu}, Yi \textsc{Hu}, Ming-Cheng \textsc{Chen}, Xiao \textsc{Jiang}, Hao \textsc{Li}, Li \textsc{Li}, Yuxuan \textsc{Li}, Nai-Le \textsc{Liu}, Chao-Yang \textsc{Lu}, Yi-Han \textsc{Luo}, Jian-Wei \textsc{Pan}, Li-Chao \textsc{Peng}, Jian \textsc{Qin}, Hui \textsc{Wang}, Zhen \textsc{Wang}, Zhen \textsc{Wang}, Guangwen \textsc{Yang}, Lixing \textsc{You}, Han-Sen \textsc{Zhong}:\textit{Quantum computational advantage using photons.} Science Magazine. 2020. Dostupné z: \url{https://science.sciencemag.org/content/370/6523/1460.full}

\bibitem{Delfs}
\textsc{Delfs}, Christina a Steven D. \textsc{Galbraith}: \textit{Computing isogenies between super-singular elliptic curves over} $\mathbb{F}_p$. Des. Codes Cryptography, 78(2), 2016. Dostupné z: \url{https://arxiv.org/abs/1310.7789}.

\bibitem{Deuring}
\textsc{Deuring}, Max: \textit{Die typen der multiplikatorenringe elliptischer funktionenkörper}. Abhandlungen aus dem mathematischen Seminar der Universität Hamburg 14, 1941. 

\bibitem{Diffie}
\textsc{Diffie}, Whitfield a Martin \textsc{Hellman}: \textit{New Directions in Cryptography}. IEEE Transactions on Information Theory 22, 1976.

\bibitem{Petit}
\textsc{Eisentr{\"a}ger}, Sean H., Kristin \textsc{Lauter}, Travis \textsc{Morrison} a Christopher \textsc{Petit}: \textit{Supersingular Isogeny Graphs and Endomorphism Rings: Reductions and Solutions.}
Advances in Cryptology – EUROCRYPT 2018, Lecture Notes in Computer Science, pages 329–368. Springer International Publishing, 2018.

\bibitem{Feynman}
\textsc{Feynman}, Richard P.: \textit{Simulating physics with computers}. Int J Theor Phys 21, 467–488, 1982. Dostupné z: \url{https://doi.org/10.1007/BF02650179}.

\bibitem{Galbraith}
\textsc{Galbraith}, Steven D.: \textit{Constructing Isogenies Between Elliptic Curves Over Finite Fields}. LMS J. Comput. Math., 199, 118-138, 1999. Dostupné z: \url{https://www.math.auckland.ac.nz/~sgal018/iso.pdf}.

\bibitem{Galbraith2}
\textsc{Galbraith}, Steven D., Florian \textsc{Hess} a Nigel P. \textsc{Smart}: \textit{Extending the GHS Weil descent attack.} EUROCRYPT 2002,  Springer LNCS 2332 29-44, 2002.

\bibitem{Galbraith3}
\textsc{Galbraith}, Steven D. a Anton \textsc{Stolbunov}: \textit{Improved Algorithm for the Isogeny Problem for Ordinary Elliptic Curves}. Applicable Algebra in Engineering, Communication and Computing, Vol. 24, No. 2, 2013. Dostupné z: \url{https://arxiv.org/abs/1105.6331}.

\bibitem{Galbraith4}
\textsc{Galbraith}, Steven D., Christopher \textsc{Petit}, Barak \textsc{Shani} a Yan \textsc{Bo Ti}: \textit{On the security of supersingular isogeny cryptosystems}. International Conference on the Theory and Application of Cryptology and Information Security. Springer, 2016.

\bibitem{Griffiths}
\textsc{Griffiths}, Robert B.: \textit{Hilbert Space Quantum Mechanics}. 2014.

\bibitem{Grover}
\textsc{Grover}, Lov K.: \textit{A fast quantum mechanical algorithm for database search}.28th Annual ACM Symposium on the Theory of Computing, 1996. Dostupné z: \url{https://arxiv.org/abs/quant-ph/9605043}.

\bibitem{Hartshorne}
\textsc{Hartshorne}, Robin: \textit{Algebraic  Geometry}. Berkley: Springer-Verlag, 1977.

\bibitem{Ireland}
\textsc{Ireland}, Kenneth a Michael \textsc{Rosen}: \textit{A Classical Introduction to Modern Number Theory}. New York, Berlin a Heidelberg: Springer-Verlag, 1982.

\bibitem{Jao}
\textsc{Jao}, David a David \textsc{Urbanik}: \textit{Extra Secrets from Automorphisms and SIDH-based NIKE}, 2018.

\bibitem{ECDSA}
\textsc{Johnson}, Don, Alfred \textsc{Menenzes} a Scott \textsc{Vanstone}: \textit{The Elliptic Curve Digital Signature Algorithm (ECDSA)}. Certicom a Department of Combinatorics \& Optimization, University of Waterloo,  Ontario, Canada. 2001.

\bibitem{Johnson}
\textsc{Johnson}, Lee W., Ronald Dean \textsc{Riess} a Jimmy Thomas \textsc{Arnold}: \textit{Introduction to Linear Algebra}. Fifth edition. Virginia Polytechnic Institute and State University: Addison-Wesley, 2002.

\bibitem{Karamlou}
\textsc{Karamlou}, Amir H, Willieam A. \textsc{Simon}, Amara \textsc{Katabarwa}, Travis L. \textsc{Scholten}, Borja \textsc{Peropandre} a Yudong \textsc{Cao}: \textit{Analyzing the Performance of Variational Quantum Factoring on a Superconducting Quantum Processor}. Zapata Computing, Boston; Research Laboratory of Electronics, Massachusetts Institute of Technology, Cambridge a IBM Quantum, IBM T. J. Watson Research Center, New York, 2020. Dostupné z: \url{https://www.zapatacomputing.com/publications/analyzing-the-performance-of-variational-quantum-factoring-on-a-superconducting-quantum-processor/}.

\bibitem{Karaskova}
\textsc{Karásková}, Zdislava: \textit{Supersingulární isogenie a jejich využití v kryptografii}. Diplomová práce. Brno: Masarykova univerzita, 2019. Dostupné z: \url{https://is.muni.cz/th/mt87i/}.

\bibitem{Koblitz}
\textsc{Koblitz}, Neal: \textit{Elliptic curve cryptosystems}. Mathematics of Computation. 48 (177): 203–209, 1987.

\bibitem{Kohel}
\textsc{Kohel}, David R.: \textit{Endomorphism rings of elliptic curves over finite fields}. University of California, Berkley, 1996.

\bibitem{Lagarias}
\textsc{Lagarias}, Jeffrey C. a Andrew M. \textsc{Odlyzko}: \textit{Effective Versions of the Chebotarev Density Theorem}. Algebraic Number Fields,L-Functions and Galois Properties (A. Fröhlich, ed.), pp. 409–464. New York, London: Academic Press, 1977.

\bibitem{Leonardi}
\textsc{Leonardi}, Christopher: \textit{A Note on the Ending Elliptic Curve in SIDH}. 2020. Dostupné z: \url{https://eprint.iacr.org/2020/262}.

\bibitem{Marcus}
\textsc{Marcus}, Daniel A.: \textit{Number fields}. New York: Springer-Verlag, 1977.

\bibitem{Matushak}
\textsc{Matushak}, Andy a Michael A. \textsc{Nielsen}: \textit{Quantum computing for the very curious}. San Francisco, 2019. Dostupné z: \url{https://quantum.country/qcvc}.

\bibitem{MOV}
\textsc{Menezes}, Afred, Tatsuki \textsc{Okamoto} a Scott \textsc{Vanstone}: \textit{Reducing Elliptic Curve Logarithms to Logarithms in a Finite Field}. IEEE Transactions on Information Theory 39, 1993.

\bibitem{Miller}
\textsc{Miller}, Victor: \textit{Use of elliptic curves in cryptography}. Advances in Cryptology—CRYPTO ’85, Lecture Notes in Computer Science, vol 218. Springer, pp 417–426, 1986.

\bibitem{Mordell}
\textsc{Mordell}, Luis J.: \textit{On the rational solutions of the indeterminate equations of the third and fourth degrees}. Cambridge, 1922.

\bibitem{Neukirch}
\textsc{Neukirch}, J{\"u}rgen: \textit{Algebraic Number Theory}. New York: Springer-Verlag, 1999.

\bibitem{NIST}
\textsc{NIST}. Post-Quantum Cryptography. Dostupné z: \url{https://csrc.nist.gov/Projects/Post-Quantum-Cryptography/}.

\bibitem{Tomas}
\textsc{Perutka}, Tomáš: \textit{Vyjadřování prvočísel kvadratickými formami.} Středoškolská odborná činnost. Brno: Masarykova univerzita, 2017. Dostupné z: \url{https://socv2.nidv.cz/archiv39/getWork/hash/ff6e75d5-f922-11e6-848a-005056bd6e49}.

\bibitem{Perutka}
\textsc{Perutka}, Tomáš: \textit{Užití dekompoziční grupy k důkazu zákona kvadratické reciprocity.} Středoškolská odborná činnost. Brno: Masarykova univerzita, 2018. Dostupné z: \url{https://socv2.nidv.cz/archiv40/getWork/hash/1984482c-1298-11e8-90e4-005056bd6e49}.


\bibitem{Pezlar}
\textsc{Pezlar}, Zdeněk: \textit{Zajímavá využití algebraické teorie čísel}. Středoškolská odborná činnost. Brno: Masarykova univerzita, 2020. Dostupné z: \url{https://socv2.nidv.cz/archiv42/getWork/hash/921aa7aa-568d-11ea-9fea-005056bd6e49}.

\bibitem{Pizer}
\textsc{Pizer}, Arnold K.: \textit{Ramanujan graphs and Hecke operators.} Bulletin of the American Math Society, 23, 1990.

\bibitem{Proos}
\textsc{Proos}, John a Christof \textsc{Zalka}: \textit{Shor’s discrete logarithm quantum algorithm for elliptic curves}. Department of Combinatorics \& Optimization, University of Waterloo,  Ontario, Canada, 2008. Dostupné z: \url{https://arxiv.org/abs/quant-ph/0301141}.

\bibitem{Pupik}
\textsc{Pupík}, Petr: \textit{Užití grupy tříd ideálů při řešení některých diofantických rovnic}. Diplomová práce. Brno: Masarykova univerzita, 2009. Dostupné z: \url{https://is.muni.cz/th/v8xsj/}.

\bibitem{Raclavsky}
\textsc{Raclavský}, Marek: \textit{Racionální body na eliptických křivkách}. Bakalářská práce. Praha: Univerzita Karlova, 2014. Dostupné z: \url{https://is.cuni.cz/webapps/zzp/detail/143352/}.

\bibitem{RSA}
\textsc{Rivest}, Ronald L., Adi \textsc{Shamir} a Leonard M. \textsc{Adleman}: \textit{A Method for Obtaining Digital Signatures and Public-Key Cryptosystems}. 1977. Dostupné z: \url{https://people.csail.mit.edu/rivest/Rsapaper.pdf}. 

\bibitem{Rosicky}
\textsc{Rosický}, Jiří: \textit{Algebra}. Brno: Masarykova univerzita, 2002.

\bibitem{Shengyu}
\textsc{Shengyu}, Zhang: \textit{Promised and Distributed Quantum Search Computing and Combinatorics}. Proceedings of the Eleventh  Annual  International Conference on Computing  and Combinatorics, Berlin, Heidelberg, 2005.

\bibitem{Shor}
\textsc{Shor}, Peter W.: \textit{Polynomial-Time Algorithms for Prime Factorization and Discrete Logarithms on a Quantum Computer}. New York: Springer-Verlag, 1994. Dostupné z: \url{https://arxiv.org/abs/quant-ph/9508027}.

\bibitem{Schoof}
\textsc{Schoof}, René: \textit{Elliptic Curves Over Finite Fields and the Computation of Square Roots $\! \operatorname{mod} \, p$.} Journal de Théorie des Nombres de Bordeaux 7, 1985. Dostupné z: \url{https://www.ams.org/journals/mcom/1985-44-170/S0025-5718-1985-0777280-6/S0025-5718-1985-0777280-6.pdf}.

\bibitem{Schoof2}
\textsc{Schoof}, René: \textit{Counting points on elliptic curves over finite fields.} Journal de Théorie des Nombres de Bordeaux 7, 1995. Dostupné z: \url{https://www.mat.uniroma2.it/~schoof/ctg.pdf}.

\bibitem{Siegel}
\textsc{Siegel}, Carl: \textit{Über die Classenzahl quadratischer Zahlkörp}. Acta Arithmetica, 1(1), 1935.

\bibitem{Silverman}
\textsc{Silverman}, Joseph H.: \textit{The Arithmetic of Elliptic Curves}. New York: Springer-Verlag, 1992. 

\bibitem{Silverman2}
\textsc{Silverman}, Joseph H.: \textit{Advanced Topics in the Arithmetic of Elliptic Curves}. New York: Springer-Verlag, 1994. 

\bibitem{Stolbunov}
\textsc{Rostovtsev}, Alexander a Anton \textsc{Stolbnov}:\textit{ Public-key cryptosystem based on isogenies}. 2006. Dostupné z: \url{http://eprint.iacr.org/2006/145/}. 


\bibitem{Suchanek}
\textsc{Suchánek}, Vojtěch: \textit{Vulkány isogenií v kryptografii}. Diplomová práce. Brno: Masarykova univerzita, 2020. Dostupné z: \url{https://is.muni.cz/th/pxawb/}.

\bibitem{Sutherland2}
\textsc{Sutherland}, Andrew V.: \textit{Isogeny Volcanoes}. 2012. Dostupné z: \url{https://arxiv.org/abs/1208.5370}.

\bibitem{Sutherland3}
\textsc{Sutherland}, Andrew V.: \textit{Identifying supersingular elliptic curves}. 2012. Dostupné z: \url{https://arxiv.org/abs/1107.1140}

\bibitem{Sutherland}
\textsc{Sutherland}, Andrew V.: \textit{Elliptic Curves}. Massachusetts Institute of Technology, 2017. Dostupné z: \url{https://math.mit.edu/classes/18.783/2017/lectures.html}. 

\bibitem{Tani}
\textsc{Tani}, Seiichiro: \textit{Claw Finding Algorithms Using Quantum Walk}. Theoretical Computer Science, 410(50):5285-5297, 2009.

\bibitem{Tate}
\textsc{Tate}, John: \textit{Endomorphisms of Abelian Varieties over Finite Fields}. Inventiones Mathematicae, 2 (2): 134–144, Cambridge, 1966.

\bibitem{Velu}
\textsc{Vélu}, Jacques: \textit{Isogénies entre courbes elliptiques}. Comptes Rendus de l’Académie des Sci-ences de Paris, 1971. 

\bibitem{Washington}
\textsc{Washington}, Lawrence C.: \textit{Elliptic Curves: Number theory and cryptography}. Maryland, 2008. 

\bibitem{Waterhouse}
\textsc{Waterhouse}, William C.: \textit{Abelian varieties over finite fields}. Annales scientifiques de l’École Normale Supérieure, 1969.

\bibitem{Weil}
\textsc{Weil}, André: \textit{L'arithmétique sur les courbes algébriques}.  Acta Mathematica 52, 1929. 


\end{thebibliography}
\end{document}



%\begin{veta} Nechť $p,q$ jsou různá lichá prvočísla. Potom 
%$$\left( \frac{p}{q} \right) = \left( \frac{q}{p} \right) \cdot (-1)^{\frac{(p-1)(q-1)}{4}}.$$

%Dále navíc Pro libovolná celá čísla $a,b$ a liché prvočíslo $p$ platí:
%\begin{enumerate}
%\item $\bigl( \frac{a}{p} \bigr)\cdot\bigl( \frac{b}{p} \bigr)=\bigl( %\frac{ab}{p} \bigr),$
%\item $\bigl( \frac{-1}{p} \bigr) = (-1)^{\frac{p-1}{2}},$
%\item $\bigl( \frac{2}{p} \bigr) = (-1)^{\frac{p^2-1}{8}}.$ 
%\end{enumerate}
%\end{veta}

%\ Vzhledem k důležitosti těchto tvrzení uvedeme ještě ekvivalentní formu, jíž je možné některé z nich vyjádřit -- a to pomocí kongruencí:

%\begin{veta} Nechť $p,q$ jsou různá lichá prvočísla. Potom 
%$$\left( \frac{p}{q} \right) = \begin{cases}
%\left( \frac{q}{p} \right) \qquad \text{pokud} \; p\;\text{nebo}\; q\equiv 1\pmod4;\\ 
%-\left( \frac{q}{p} \right) \qquad \mbox{pokud} \; p\equiv q\equiv 3 \pmod{4}. \end{cases} $$

%Dále navíc pro libovolná celá čísla $a,b$ a liché prvočíslo $p$ platí:
%\begin{enumerate}
%\item $\bigl( \frac{a}{p} \bigr)\cdot\bigl( \frac{b}{p} \bigr)=\bigl( \frac{ab}{p} \bigr),$
%\item $\bigl( \frac{-1}{p} \bigr) =
%\begin{cases}
%1 \quad \text{pokud} \; p\equiv1\pmod4\\
%-1\quad\text{pokud}\; p\equiv3\pmod4
%\end{cases}$
%\item $\bigl( \frac{2}{p} \bigr) = 
%\begin{cases}
%1 \quad \text{pokud} \; p\equiv\pm1\pmod8\\
%-1\quad\text{pokud}\; p\equiv\pm3\pmod8.
%\end{cases}$
%\end{enumerate}
%\end{veta


% Cvičení: falešný násobení PO_L


% \Zdroje: Dumit Foote, Zakony reciprocity, Rosicky, Cox, Marcus, clanek o Mihaelescau?, Ireland Rosen?, Pupik?

%\begin{definice} Množinu $G$ spolu s binární operací $\odot$ na ní definovanou nazveme grupou, pokud splňuje tyto podmínky:
%\begin{enumerate}
%\item operace je asociativní, tzn.\ pro každé $x,y,z \in G$ platí $(x\odot y)\odot z=x\odot(y\odot z)$,
%\item existuje tzv.\ neutrální prvek, tedy nějaké $e \in G$ takové, že pro každé $x \in G$ platí $e\odot x=x=x\odot e$,
%\item ke každému prvku můžeme nalézt prvek k němu inverzní, tedy pro každé $x\in G$ existuje $y\in G$ tak, že $x\odot y=e=y\odot x$. 
%\end{enumerate}
%\ Pokud je operace navíc komutativní, hovoříme o abelovské nebo komutativní grupě.
%\end{definice}

%\begin{definice} Nechť R je množina, $+$, $\cdot$ binární operace na ní definované. Pak $(R,+,\cdot)$ je okruh, pokud:
%\begin{enumerate}
%\item $(R,+)$ je komutativní grupa,
%\item operace $\cdot$ je asociativní a existuje vzhledem k ní neutrální prvek,
%\item platí oboustranná distributivita, tedy pro libovolné $a,b,c\in R$ platí $a\cdot(b+c)=a\cdot b+a\cdot c,$ $(b+c)\cdot a=b\cdot a+ c\cdot a.$
%\end{enumerate}
%\ Pokud je i operace $\cdot$ komutativní, hovoříme o komutativním okruhu.
%\end{definice}

%\ Operaci + běžně nazýváme sčítání a neutrální prvek vůči ní značíme symbolem 0, operaci $\cdot$ nazýváme násobení a neutrální prvek vůči ní značíme jako $1$.



%\begin{definice} Komutativní okruh $R$ nazýváme obor integrity, pokud pro libovolná $a,b\in R$ platí, že pokud $a\cdot b=0$, tak $a=0$ nebo $b=0$. \end{definice}


%\begin{definice} Nechť T je množina, $+$, $\cdot$ binární operace na ní definované. Pak $(T,+,\cdot)$ je těleso, pokud:
%\begin{enumerate}
%\item $(T,+)$ je komutativní grupa,
%\item $(T\smallsetminus\{0\},\cdot)$ je komutativní grupa,
%\item platí oboustranná distributivita, tedy pro libovolné $a,b,c\in T$ platí $a\cdot(b+c)=a\cdot b+a\cdot c,$ $(b+c)\cdot a=b\cdot a+ c\cdot a.$
%\end{enumerate}
%\end{definice}

%\ Jinak řečeno, těleso je takový obor integrity, jehož každý nenulový prvek je jednotkou, neboli je \textit{invertibilní} -- tedy má vůči operaci $\cdot$ inverzní prvek.






%V případě $p=2$ se nám situace ztíží tím, že pokud $m\equiv1\pmod4$, nemůžeme aplikovat větu \ref{polynomy}, jelikož 2 dělí $|\z[\frac{1+\sqrt m}2]/\z[\sqrt m]|$. Přesto ale dokážeme následující větu:

%\begin{veta} Nechť $K=\q(\sqrt m)$. Potom: $$2\o_K=
%\begin{cases}
%(2,\sqrt m)^2 \quad \text{pokud}\; m\equiv2\pmod4, \\
%(2,1+\sqrt m)^2 \quad \text{pokud} \;m\equiv3\pmod4, \\
%(2,\frac{1-\sqrt m}2)(2,\frac{1+\sqrt m}2) \quad \text{pokud}\; m\equiv1\pmod 8, \\
%2\o_K,\; \text{tj. je prvoideál} \quad \text{pokud}\; m\equiv 5\pmod 8.
%\end{cases}$$
%\end{veta}

%\begin{proof} Zamysleme se nejprve nad diskriminantem okruhu $\o_K$. Z věty \ref{tabulka} víme, že v případě $m\equiv 2,3\pmod4$ platí $d(\o_K)=4m$ a v případě $m\equiv1\pmod4$ platí $d(\o_K)=m$.

%\ Uvažujme nejprve $m\equiv2,3\pmod4$. V tomto případě můžeme aplikovat větu \ref{polynomy}. Jelikož $d(\o_K)=4m$, tak se 2 bude vždy větvit.

%\ Pokud $m\equiv2\pmod4$, tak $2|m$ a tedy $x^2-m\equiv(x)^2\pmod2$. Tudíž $2\o_K=(2,\sqrt m)^2$ (analogicky jsme postupovali v důkazu předchozí věty).

%\ Pokud $m\equiv3\pmod4$, tak jelikož se 2 větví a zároveň nedělí $m$, platí $x^2-m\equiv x^2+1\equiv x^2+2x+1\equiv (x+1)^2\pmod2$ a tedy $2\o_K=(2,1+\sqrt m)^2$.

%\ Nyní uvažujme $m\equiv1\pmod4$, tedy $\o_K=\z[\frac{1+\sqrt m}2]$. Víme již, že nemůžeme použít větu \ref{polynomy}, musíme tedy použít jiné argumenty.

%\ Pokud $m\equiv1\pmod 8$, tak $2\in (2,\frac{1-\sqrt m}2)(2,\frac{1+\sqrt m}2)=(4,1+\sqrt m,1-\sqrt m,\frac{1- m}4)$ protože $\nsd(4,\frac{1-m}4)=2$ (a díky Bezoutově rovnosti s každými dvěma celými čísly ležícími v daném ideálu v něm leží i jejich největší společný dělitel). Tudíž $2\o_K\s(2,\frac{1-\sqrt m}2)(2,\frac{1+\sqrt m}2)$ a proto $(2,\frac{1-\sqrt m}2)(2,\frac{1+\sqrt m}2)|2\o_K$. Aby platila věta \ref{eifi}, musí už platit přímo rovnost.

%\ Zbývá případ $m\equiv 5\pmod 8$. Nechť $\P$ je prvoideál $o\_K$, $\P|2\o_K$. Ukážeme $f(\P|2)=2$. To uděláme sporem: pokud $f(\P|2)=1$, tak $\o_K/\P\cong\z/2\z$. Uvažujme polynom $x^2-x+\frac{1-m}4$. Ten má v $\o_K$ kořen $\frac{1+\sqrt m}4$; má tedy kořen i v $\o_K/\P$. Na druhou stranu tento polynom v $\z/2\z$ žádný kořen nemá, jelikož $x^2-x+\frac{1-m}4\equiv x^2-x+1\pmod2$. To je spor s tím, že jsou tělesa $\o_K/\P$ a $\z/2\z$ izomorfní a $2\o_K$ je tedy opravdu prvoideál.

%\
%\end{proof}

%\ Případ $p=2$ nás zajímá spíše pro úplnost, případ $p$ je liché bude hrát ústřední roli v důkazu kvadratické recprocity a dalších tvrzení. 





\begin{poznamka} Často nastává situace, kdy $K$ je \uv{skoro} podtěleso $L$. Uvažujme například těleso $\R$ s klasickým sčítáním a násobením a těleso $\R^2=\{(a,b)\mid (a,b)\in\R\}$ s operacemi sčítání po složkách (tj. $(a,b)+(c,d)=(a+c,b+d)$) a s násobením definovaným jako $(a,b)\cdot(c,d)=(ac-bd,ad+bc)$ pro všechna $a,b,c,d\in\R$.Sice $\R$ není podtělesem tělesa $\R^2$, ale existuje vnoření (tj. injektivní homomorfismus) $f: \R\rightarrow\R^2$ (např. definované jako $f(a)=(a,0)$), tzn. $\R$ je izomorfní s nějakým podtělesem $f(\R)=\{f(a)\mid a\in\R\}$ tělesa $\R^2$. Sice tedy přísně vzato nemůžeme hovořit o rozšíření $\R\s\R^2$, ale jelikož $\R$ a $f(\R)$ jsou izomorfní, tudíž mají s algebraického hlediska identické vlastnosti, tak někdy nebudeme zcela korektní a např. v této situaci budeme mluvit o rozšíření $\R\s\R^2$ místo o $f(\R)\s\R^2$. \label{nejsmekorektni} \end{poznamka}

\ Zohledníme-li tuto poznámku, můžeme psát $[\R^2:\R]=2$.

\section{Základní poznatky}

\ V této části stručně připomeneme pojmy z algebry, které budeme v práci nejčastěji používat -- především hlavní větu o faktorgrupách, ideál okruhu a vlastnosti okruhu polynomů jedné proměnné.

\ Uveďme tedy nejprve hlavní větu o faktorgrupách:

\begin{veta} Nechť $f:G\rightarrow K$ je homomorfismus grup, $H$ normální podgrupa grupy $G$ splňující $H\s\ker f$. Nechť $\pi:G\rightarrow G/H$ je projekce grupy $G$ na faktorgrupu $G/H$. Pak existuje, a to jediné, zobrazení $\fii:G/H\rightarrow K$ splňující $\fii\odot\pi=f$. Navíc platí: \begin{enumerate}
\item $\fii$ je homomorfismus grup,
\item $\fii$ je injekce, právě když $H=\ker f$,
\item $\fii$ je surjekce, právě když $f$ je surjekce. \end{enumerate} \end{veta}
$$
\xymatrix{
G\ar[rr]^f\ar[dr]^\pi&&K \\
&G/H\ar@{-->}[ur]_\fii&
}
$$

\ Věta má podstatné důsledky:

\begin{dusledek} Nechť $f:G\rightarrow K$ je homomorfismus grup, $f(G)=\{f(g)\mid g\in G\}$. Pak $G/\ker f\cong f(G)$. \end{dusledek}

\begin{dusledek} Homomorfismus grup $f:G\rightarrow K$ je injektivní, právě když $\ker f$ je triviální grupa. \end{dusledek}

Nyní přejděme k pojmu ideál.
 

\begin{poznamka} V dalším textu budeme používat následující značení: pro těleso $K$ symbolem $K(a_1,...,a_n)$ míníme těleso generované množinou $K\cup\{a_1,...,a_n\}$. V případě okruhů používáme obdobné značení -- nejmenší okruh obsahující nějaký okruh $R$ a množinu $\{a_1,...,a_n\}$ značíme $R[a_1,...,a_n]$. Tedy např. $\mathbb{Q}(\sqrt 2)$ nejmenší těleso obsahující racionální čísla a odmocninu ze dvou a $\z[i]$ je nejmenší okruh obsahující celá čísla a imaginární jednotku~$i$. \end{poznamka}


\begin{definice} Nechť R je okruh. Neprázdnou množinu $\I\subseteq R$ nazveme ideálem okruhu R, pokud:
\begin{enumerate}
\item pro libovolné $a,b\in \I$ platí $a+b\in \I$,
\item pro libovolné $r\in R, a\in \I$ platí $ar\in \I, ra\in \I$.
\end{enumerate}
\end{definice}

\ Každý okruh má alespoň dva ideály, a to celý okruh a triviální ideál $\{0\}$ -- říkáme jim nevlastní ideály a ostatní ideály nazýváme vlastní. 

\begin{definice} Ideál $\I$ okruhu $R$ nazýváme hlavní, pokud je ve tvaru $aR=\{ar|r\in R\}$ pro nějaké $a\in R$. Danému oboru integrity říkáme okruh hlavních ideálů, pokud je každý jeho ideál hlavní. \end{definice}

\ Typickým okruhu hlavních ideálů jsou celá čísla: jediné ideály jsou zde tvaru $n\mathbb{Z}$, kde $n$ je libovolné nezáporné celé číslo.

\begin{poznamka} Hlavní ideál $aR$ někdy značíme $(a)$. Obecně je možné definovat ideál generovaný množinou a ideál generovaný konečnou množinou $\{a_1,a_2,...,a_n\}$ značíme $(a_1,a_2,...,a_n)$.\end{poznamka}

\ Všimněme si, že z definice ideálu přímo plyne důležitý poznatek:

\begin{veta} Nechť $\I$ je ideál okruhu R. Pak $(\I,+)$ je normální podgrupa grupy (R,+). \end{veta}

\ Ideály jsou úzce spjaté s homomorfismy okruhů. Platí totiž následující věta:

\begin{veta} Nechť $f:R\rightarrow S$ je homomorfismus okruhů. Pak platí:
\begin{enumerate}
\item je-li $\J$ ideál okruhu $S$, pak $f^{-1}(\J)=\{x\in R|f(x)\in \J\}$ je ideál okruhu $R$,
\item je-li $f$ surjekce a $\I$ ideál okruhu $R$, pak $f(\I)=\{f(x)|x\in \I\}$ je ideál okruhu $S$.
\end{enumerate}
\end{veta}

\ To mimo jiné znamená, že jádro libovolného homomorfismu $f:R\rightarrow S$ je ideál okruhu $R$, jelikož $\ker f=f^{-1}(0)$.

\ Podle ideálů můžeme faktorizovat. Jelikože je pro každý ideál $\I$ je $(\I,+)$ normální podgrupa $(R,+)$, můžeme sestrojit faktorgrupu $(R/\I,+)$, kde + je nyní sčítání tříd pomocí reprezentantů. Lze ukázat, že na této faktorgrupě je možné definovat i násobení pomocí reprezentantů tak, že $R/\I$ je s těmito operacemi okruh. Takto vzniklý okruh nazýváme faktorokruh. Existuje hlavní věta o faktoroktuzích analogická hlavní větě o faktorgrupách.

\ Existují dvě významné skupiny ideálů, které má smysl definovat:

\begin{definice} Nechť R je okruh, $\I$ jeho vlastní ideál. O ideálu $\I$ říkáme, že je:
\begin{enumerate}
\item prvoideál, pokud pro libovolné $a,b\in R$ platí implikace $ab\in \I\Rightarrow a\in \I nebo b\in \I$,
\item maximální ideál, pokud neexistuje žádný ideál $\J$ okruhu $R$ splňující $\I\subsetneq \J\subsetneq R$.
\end{enumerate}
\end{definice}

\ Pokud je okruh $R$ komutativní, můžeme tyto skupiny ideálů poznat podle toho, jak vypadá jimi určený faktorokruh:

\begin{veta} Nechť $R$ je komutativní okruh, $\I$ jeho vlastní ideál. Pak je $\I$ prvoideál, právě když faktorokruh $R/\I$ je obor integrity, a $\I$ je maximální ideál, právě když je $R/\I$ těleso. \label{prvmax} \end{veta}

\ Dále stručně připomeňme vlastnosti polynomů jedné proměnné.

\begin{veta} Nechť $R$ je okruh. Pak množina $R[x]$ polynomů jedné proměnné s koeficienty z $R$ tvoří rovněž okruh (s obvykle definovaným sčítáním a násobením polynomů). Navíc je-li $R$ komutativní (resp. obor integrity), tak i $R[x]$ je komutativní (resp. obor integrity). \end{veta}

\begin{veta} Nechť $R$ je obor integrity. Pak polynom $f\in R[x]$ má v $R$ nejvýše tolik kořenů, kolik je jeho stupeň (který značíme $\deg f$). \end{veta}

\begin{veta} Nechť $R$ je obor integrity. Pak je $R[x]$ euklidovský okruh, přesněji pro každé dva polynomy $f,g\in R[x]$ existují právě jedna dvojice polynomů $k,r\in R[x]$ taková, že $f=kg+r$ a $\deg r<\deg g$. \end{veta} 

\ Na závěr definujme podílové těleso a uveďme některé jeho příklady.

\begin{definice} Nechť $R$ je obor integrity. Nejmenší těleso obsahující $R$ nazveme podílové těleso okruhu R. \end{definice}

\begin{veta} Nechť $R$ je obor integrity. Pak jeho podílové těleso $F$ můžeme psát ve tvaru $$F=\left\{\frac ab\mid a,b\in R, b\ne0\right\},$$ kde $\frac ab=\frac cd$, právě když $ad=bc$ a operace sčítání a násobení provádíme následovně: $$\frac ab+\frac cd=\frac{ad+bc}{bd},$$ $$\frac ab\cdot \frac cd= \frac{ac}{bd}.$$ \end{veta}

\ Podílové těleso okruhu $\z$ je těleso racionálních čísel. Podílové těleso okruhu $R[x]$ nazýváme \textit{těleso racionálních funkcí} a značíme ho $R(x)$. Tyto pojmy můžeme zobecnit: množina polynomů více proměnných nad oborem integrity $R$ tvoří rovněž obor integrity a jeho podílové těleso také nazýváme těleso racionálních funkcí.
